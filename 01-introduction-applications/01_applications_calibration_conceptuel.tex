\subsection{Aperçu de l’invariant \(I_{1}(T)=P(T)/T\)}

Pour visualiser la dynamique de l’invariant \(I_{1}(T)\), calculé à partir de l’intégration de \(\dot P(T)\) (grille~v3), on trace \(I_{1}(T)\) en échelle \emph{log–log}, comparée à la valeur unité du modèle standard \(\Lambda\mathrm{CDM}\).

Le script \texttt{zz-scripts/chapter01/trace\_fig02.py} importe la table dense
\texttt{zz-data/chapter01/01\_P\_en\_fonction\_de\_T.dat} et calcule
l’invariant \(I_{1}(T)=P(T)/T\) avant de générer la figure suivante :

\begin{figure}[htbp]
  \centering
  \includegraphics{zz-figures/chapter01/fig_02_calibration_logistique.png}
  \caption{Fig.~02 – Évolution de l’invariant \(I_{1}(T)=P(T)/T\) (échelle log–log, grille~v3) :
           MCGT : \(I_{1}=P/T\) et référence \(\,I_{1}=1\) (ligne pointillée).}
  \label{fig:ratio_PT_log}
\end{figure}

Ce tracé met en évidence :
\begin{itemize}
  \item un pic initial modéré dû à la transition logistique et à l’intégration de \(\dot P(T)\),
  \item une stabilisation vers la valeur unité (\(I_{1}=1\)) pour \(T\gg T_{c}\),
\end{itemize}
validant ainsi la cohérence physique et la convergence de l’invariant sous l’effet du paramétrage logistique optimisé.

Ce pic initial modéré de \(I_{1}(T)\) traduit la même dynamique de croissance rapide
qui conduit à la formation précoce des galaxies massives (cf. section suivante).

\subsection{Formation précoce des galaxies (JWST, \(T=0.30\) Gyr)}

Dans cet exemple de référence (jeu historique), on adopte les paramètres suivants pour illustrer l’impact du paramétrage logistique « à l’ancienne » sur la croissance des fluctuations :
\[
  \alpha_{0}=0.40,\quad
  \alpha_{\infty}=1.00,\quad
  T_{c}=0.25\,\mathrm{Gyr},\quad
  \Delta=0.10\,\mathrm{Gyr},\quad
  T_{p}=0.02\,\mathrm{Gyr}.
\]
On calcule alors \(\alpha_{\mathrm{log}}(T)\) et \(\alpha(T)\) comme en section~\ref{sec:cadre_theorique}, puis on intègre la dérivée complète :
\[
  \dot P(T)
    = \alpha(T)\,T^{\alpha(T)-1}
    + T^{\alpha(T)}\,\ln T\,\frac{d\alpha}{dT},
\qquad
  P(T)
    = 1 + \int_{T_{0}}^{T}\dot P(t)\,\mathrm{d}t,
\quad
  T_{0}=10^{-6}\,\mathrm{Gyr}.
\]
L’invariant s’écrit
\[
  I_{1}(T)=\frac{P(T)}{T},
\]
et son évaluation à \(T=0.30\) Gyr, obtenue à partir des données
\texttt{zz-data/chapter01/01\_chronologie\_resultats.csv}, donne
\[
  I_{1}(0.30)\approx1.32.
\]
Cette valeur, bien qu’inférieure à l’estimation naïve \(0.30^{\alpha-1}\) (correspondant à l’hypothèse simplificatrice \(\dot P(T)=0\)), reste suffisamment élevée pour expliquer la formation rapide de galaxies massives (\(M_{\star}>10^{11}\,M_{\odot}\)) à \(z\gtrsim13\) observée par le JWST (JADES/CEERS).

Pour une comparaison entre ce jeu historique et la calibration optimisée v3, voir Chapitre 4.

\subsection{Correction de la tension sur \(H_{0}\) à la recombinaison}

Au moment de la recombinaison (\(T_{\mathrm{rec}}\approx3.8\times10^{-4}\) Gyr),
puisque \(H\propto P/a\), l’intégration de \(\dot P(T)\) dans la configuration optimisée induit une légère correction de l’expansion précoce :
\[
  \frac{\delta H_{0}}{H_{0}}
  =\Bigl.\frac{\partial\ln(P/a)}{\partial\ln a}\Bigr|_{z_{\rm rec}}
  =\Bigl.\frac{\partial\ln P}{\partial\ln a}\Bigr|_{z_{\rm rec}} -1
  \;\simeq\;9.2\times10^{-4},
  \quad
  a=(1+z)^{-1}\;(\text{où }z_{\rm rec}\approx1100).
\]
Les valeurs mesurées sont
\[
  H_{0}^{\rm loc}=73.0\pm1.0\;\mathrm{km\,s^{-1}\,Mpc^{-1}},
  \quad
  H_{0}^{\rm CMB}=67.4\pm0.5\;\mathrm{km\,s^{-1}\,Mpc^{-1}},
\]
soit un écart initial d’environ \(6\%\), ramené à
\[
  \frac{\delta H_{0}}{H_{0}}\simeq9.2\times10^{-4}
  \quad(\text{soit }0.092\%).
\]

Les données numériques exactes figurent dans
\texttt{zz-data/chapter01/01\_H0\_tension\_MCGT.csv}.

Les écarts relatifs \(\varepsilon_i\) aux neuf jalons sont détaillés dans
\texttt{zz-data/chapter01/01\_ecart\_relatif\_chronologie.csv}.

Le script \texttt{zz-scripts/chapter01/trace\_fig03.py} importe ces données
et génère la figure suivante :

\begin{figure}[htbp]
  \centering
  \includegraphics[width=0.75\linewidth]{zz-figures/chapter01/fig_03_ecart_rel_chronologie.png}
  \caption{Fig.~03 – Écart relatif
    \(\varepsilon(T)
      =\frac{|P_{\rm calc}(T)-P_{\rm ref}(T)|}{P_{\rm ref}(T)}\times100\%\)
    aux neuf âges clés (grille~v3, configuration optimisée) : tous les écarts
    restent \(<1\%\).}
  \label{fig:ecart_relatif_chronologie}
\end{figure}

Au-delà de cette correction relative de la recombinaison, la dynamique des invariants confirme la robustesse du paramétrage logistique optimisé.

\subsection{Invariants adimensionnels}

Le MCGT définit trois invariants :
\[
  I_{1}(T)=\frac{P(T)}{T},\quad
  I_{2}(T)=\frac{\dot P(T)}{P(T)},\quad
  I_{3}(T)=\frac{T\,\dot P(T)}{P(T)},
\]
qui rendent compte de l’évolution relative de la fonction propre du temps.
Dans la configuration optimisée (grille~v3) on observe :
\[
  I_{1}(T)\;\text{pic modéré à }\mathcal{O}(10)\text{ autour de }T_{c},
  \quad
  I_{2}(T),\,I_{3}(T)\ll1
  \quad
  \text{pour }T\in[10^{-6},14]\;\mathrm{Gyr}.
\]
Pour une visualisation détaillée en échelle log–log et des histogrammes des distributions, voir la Fig.~03 au Chapitre 4 (Invariants adimensionnels) et les données dans
\texttt{zz-data/chapter01/01\_invariants_adimensionnels.csv}.

\subsection{Calibration logistique détaillée}

\subsubsection{Objectif et données d’entrée}

L’objectif est de reproduire exactement les neuf jalons
\((T_i,P_{\mathrm{ref}}(T_i))\) (cf. Chapitre 2) en ajustant la fonction propre
du temps \(P(T)\) pour que
\[
  P_{\mathrm{calc}}(T_{i}) = P_{\mathrm{ref}}(T_{i}),
  \quad i=1,\dots,9.
\]

Les données d’entrée sont :
\begin{itemize}
  \item \texttt{zz-data/chapter01/01\_chronologie\_resultats.csv} :
        neuf paires \((T_i,P_{\mathrm{ref}})\).
  \item \texttt{zz-data/chapter01/01\_P\_en\_fonction\_de\_T.dat} :
        table dense \((T,P_{\mathrm{calc}}(T))\) sur \(T\in[10^{-6},14]\) Gyr.
\end{itemize}

\subsubsection{Paramètres considérés}
On conserve la forme paramétrique v3 (section~\ref{sec:cadre_theorique}),
avec paramètres libres :
\[
  \alpha_{0},\quad T_{c},\quad \Delta,\quad \frac{T_{p}}{T_{c}}\quad\bigl(T_{p}/T_{c}=0.14\bigr).
\]

\begin{table}[htbp]
  \centering
  \begin{tabular}{l S S S S}
    \toprule
     & {$\alpha_{0}$} & {$T_{c}$ (Gyr)} & {$\Delta$ (Gyr)} & {$T_{p}/T_{c}$} \\
    \midrule
    Configuration historique   & 0.40  & 0.25   & 0.10   & 0.08 \\
    Configuration optimisée~(v3) & 0.72  & 0.62   & 0.034  & 0.14 \\
    \bottomrule
  \end{tabular}
  \caption{Comparaison des jeux de paramètres historique et optimisé (grille~v3).}
  \label{tab:parametres_calibration}
\end{table}

La configuration optimisée garantit que, pour tout jalon \(T_{i}\) :
\[
  \bigl|P_{\mathrm{calc}}(T_{i}) - P_{\mathrm{ref}}(T_{i})\bigr|
  < 0.01\,P_{\mathrm{ref}}(T_{i}),
\]
c’est-à-dire un écart relatif \(<1\%\).

\subsubsection{Méthode de minimisation}

Le script \texttt{zz-scripts/chapter01/integration\_chronologie.py} suit ces étapes :
\begin{enumerate}
  \item Lecture des neuf paires \texttt{zz-data/chapter01/01\_chronologie\_resultats.csv}.
  \item Interpolation \emph{log–log} de la table brute
      \texttt{zz-data/chapter01/01\_P\_en\_fonction\_de\_T.dat}
      (le tracé de la Fig.~\ref{fig:p_vs_t_calibration} est en échelle log–lin).
  \item Pour chaque quadruplet
        \((\alpha_{0},T_{c},\Delta,T_{p}/T_{c})\) de la grille testée :
    \begin{itemize}
      \item Calcul de \(\alpha(T)\) et \(P_{\mathrm{calc}}(T)\) par intégration de \(\dot P\).
      \item Évaluation de
      \[
        \chi^{2}
        =
        \sum_{i=1}^{9}
        \bigl[P_{\mathrm{calc}}(T_{i}) - P_{\mathrm{ref}}(T_{i})\bigr]^{2}
        \quad(\text{sans pondération}).
      \]
    \end{itemize}
  \item Sélection du jeu de paramètres minimisant \(\chi^{2}\).
  \item Sauvegarde du résultat optimal et génération des figures de référence.
\end{enumerate}

\subsubsection{Figures de calibration}

Les sources pour ces tracés sont :
\begin{itemize}
  \item Données :
    \texttt{zz-data/chapter01/01\_chronologie\_resultats.csv},
    \texttt{zz-data/chapter01/01\_P\_en\_fonction\_de\_T.dat},
    \texttt{zz-data/chapter01/01\_P\_initial.csv},
    \texttt{zz-data/chapter01/01\_P\_optimise.csv},
    \texttt{zz-data/chapter01/01\_P\_initial\_calc\_en\_fonction\_de\_T.dat},
    \texttt{zz-data/chapter01/01\_P\_optimise\_calc\_en\_fonction\_de\_T.dat},
    \texttt{zz-data/chapter01/01\_deriveeP\_initiale.csv}.
  \item Scripts :
    \texttt{zz-scripts/chapter01/integration\_chronologie.py},
    \texttt{zz-scripts/chapter01/trace\_fig04.py},
    \texttt{zz-scripts/chapter01/trace\_fig05.py}.
\end{itemize}

\begin{figure}[htbp]
  \centering
  \includegraphics[width=0.75\linewidth]{zz-figures/chapter01/fig_04_evolution_P_en_fonction_de_T.png}
  \caption{Fig.~04 – Évolution de \(P_{\mathrm{calc}}(T)\) (trait continu, log–lin) :
           configuration optimisée (grille~v3) avec interpolation log–log sur la table brute.}
  \label{fig:p_vs_t_calibration}
\end{figure}

Cette représentation de \(P(T)\) montre clairement la progression continue de la courbe optimisée.
Le diagramme suivant met en perspective cette évolution en la confrontant directement aux neuf jalons.

\begin{figure}[htbp]
  \centering
  \includegraphics[width=0.75\linewidth]{zz-figures/chapter01/fig_05_diagramme_calibration.png}
  \caption{Fig.~05 – Diagramme de calibration (grille~v3, échelle log–log) :
           configuration optimisée et points de référence aux neuf jalons.}
  \label{fig:diagramme_calibration}
\end{figure}

\subsubsection{Comparaison des dérivées temporelles}

Pour évaluer l’impact de l’optimisation sur la dérivée propre du temps \(\dot P(T)\),
nous exploitons trois jeux de données :

\begin{itemize}
  \item \texttt{zz-data/chapter01/01\_deriveeP\_initiale.csv} :
        dérivées \(\dot P\) historiques aux neuf jalons.
  \item \texttt{zz-data/chapter01/01\_deriveeP\_optimisee.csv} :
        dérivées \(\dot P\) optimisées aux neuf jalons (grille~v3).
  \item \texttt{zz-data/chapter01/01\_deriveeP\_en\_fonction\_de\_T.dat} :
        table dense \(\dot P(T)\) sur \(T\in[10^{-6},14]\) Gyr pour la configuration optimisée.
\end{itemize}

Le script \texttt{zz-scripts/chapter01/trace\_fig06.py} importe ces trois fichiers
et produit la figure suivante :

\begin{figure}[htbp]
  \centering
  \includegraphics[width=0.75\linewidth]{zz-figures/chapter01/fig_06_comparaison_deriveeP_initiale_vs_deriveeP_optimisee.png}
  \caption{Fig.~06 – Comparaison de \(\dot P(T)\) :
           dérivées aux neuf jalons pour la configuration historique (gris clair)
           et pour la configuration optimisée (orange, grille~v3), ainsi que la courbe dense optimisée.}
  \label{fig:comparaison_deriveeP}
\end{figure}

Cette comparaison met en évidence la réduction des écarts sur \(\dot P\)
au niveau de la transition logistique, confirmant l’atténuation du pic
et la cohérence physique de l’optimisation v3.

\subsection{Synthèse des données finales}

La calibration de \(P(T)\) sur les neuf jalons confirme un écart relatif
inférieur à \(1\%\), validant la cohérence de l’ajustement paramétrique optimisé.

\subsection{Glossaire}

\begin{description}
  \item[$T$] Âge de l’Univers, en milliards d’années (\(\mathrm{Gyr}\)).
  \item[$T_{0}$] Âge initial de normalisation (\(10^{-6}\)\,\(\mathrm{Gyr}\)).
  \item[$\alpha_{\mathrm{log}}(T)$] Exposant logistique pur,
    \(\displaystyle \alpha_{0}
      + \frac{1-\alpha_{0}}{1 + e^{-(T - T_{c})/\Delta}}\).
  \item[$P(T)$] Fonction propre du temps,
  normalisée à \(P(T_{0})=1\) et obtenue par intégration de \(\dot P(T)\).
  \item[Calibration logistique] Ajustement paramétrique de \(P(T)\) sur neuf jalons (grille~v3).
  \item[$\alpha(T)$] Exposant complet intégrant le plateau précoce,
    \(\displaystyle \alpha_{\mathrm{log}}(T)\bigl[1 - e^{-(T/T_{p})^{2}}\bigr]\),
    avec \(T_{p}/T_{c}=0.14\) (grille~v3).
  \item[$T_{c},\,\Delta$] Paramètres central et largeur de la transition logistique (Gyr).
  \item[$T_{p}$] Paramètre du plateau précoce (seul paramètre associé au plateau).
  \item[$\dot P(T)$] Dérivée de \(P\) :
    \[
      \dot P(T)
      = \alpha(T)\,T^{\alpha(T)-1}
      + T^{\alpha(T)}\,\ln T\,\frac{d\alpha}{dT}.
    \]
  \item[$I_{1}(T)=P(T)/T$] Invariant adimensionnel décrivant l’évolution relative de \(P(T)\) (cf. Chapitre 4 pour \(I_{2}\) et \(I_{3}\)).
  \item[$\delta H_{0}/H_{0}$] Correction relative de la constante de Hubble à la recombinaison,
    \(\displaystyle\frac{\delta H_{0}}{H_{0}}\approx9.2\times10^{-4}.\)
\end{description}
