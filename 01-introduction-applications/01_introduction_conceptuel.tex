\section{Chapitre 1 – Introduction conceptuelle}

\subsection{Cadre théorique et paramétrisation logistique}

Dans le Modèle de Courbe Gravitationnelle du Temps (MCGT), on module l’évolution cosmique du temps propre via un exposant variable \(\alpha(T)\). On définit d’abord un exposant logistique pur :
\[
  \alpha_{\mathrm{log}}(T)
  = \alpha_{0}
    + \frac{1 - \alpha_{0}}{1 + \exp\!\bigl[-(T - T_{c})/\Delta\bigr]},
\]
où :
\begin{itemize}
  \item \(\alpha_{0}\) est l’exposant initial (\(T\ll T_{c}\)),
  \item \(T_{c}\) est l’âge central de la transition,
  \item \(\Delta\) est la largeur de la zone de transition.
\end{itemize}

\begin{center}
  \fbox{\small%
    \(\displaystyle
      \alpha_{0}=0.72,\;
      T_{c}=0.62\,\mathrm{Gyr},\;
      \Delta=0.034\,\mathrm{Gyr},\;
      T_{p}=0.14\,T_{c}\simeq0.087\,\mathrm{Gyr}.
    \)
  }
\end{center}
Les valeurs proviennent de l’optimisation \texttt{least\_squares} (cf. Tab.~\ref{tab:parametres_opt}).

Pour tenir compte d’un plateau précoce, on introduit le facteur
\[
  1 - \exp\!\bigl(-(T/T_{p})^{2}\bigr),
\]
de sorte que l’exposant complet s’écrit :
\[
  \alpha(T)
  = \alpha_{\mathrm{log}}(T)\,\bigl(1 - \exp\!\bigl(-(T/T_{p})^{2}\bigr)\bigr).
\]

On normalise la fonction propre du temps par
\[
  P(T_{0}) = 1
  \quad\bigl(\text{plateau initial de normalisation, }T_{0}=10^{-6}\,\mathrm{Gyr}\bigr).
\]
La dynamique est donnée par :
\[
  \dot P
    = \alpha(T)\,T^{\alpha(T)-1}
    + T^{\alpha(T)}\,\ln T\,\frac{d\alpha}{dT},
\qquad
  P(T)
    = 1 + \int_{T_{0}}^{T}\dot P\,\mathrm{d}t,
\]
assurant la condition initiale et la convergence de l’invariant \(I_{1}(T)=P(T)/T\).

On a par ailleurs :
\[
  \frac{d\alpha}{dT}
  = \underbrace{\frac{1-\alpha_{0}}{\Delta}\,
    \frac{e^{-(T-T_{c})/\Delta}}{\bigl(1+e^{-(T-T_{c})/\Delta}\bigr)^{2}}
  }_{\text{logistique}}
  \bigl(1-e^{-(T/T_{p})^{2}}\bigr)
  + \alpha_{\mathrm{log}}(T)\,\frac{2T}{T_{p}^{2}}\,e^{-(T/T_{p})^{2}}.
\]

Les neuf jalons \((T_i,P_{\mathrm{ref}})\) employés pour l’optimisation sont récapitulés dans la Table~\ref{tab:jalons_ecarts}.
Les valeurs exactes sont dans le fichier
\texttt{zz-data/chapter01/01\_chronologie\_resultats.csv}.

Le script \texttt{zz-scripts/chapter01/trace\_fig01.py} importe ces jalons
et génère la figure suivante :

\begin{figure}[htbp]
  \centering
  \includegraphics[width=\linewidth]{zz-figures/chapter01/fig_01_plateau_precoce.png}
  \caption{Fig.~01 – Plateau précoce de \(P(T)\) (grille~v3) :
           intégration complète de \(\dot P(T)\) avec les paramètres de la Table~\ref{tab:parametres_opt};
           repère vertical à \(T_{p}\simeq0.087\,\mathrm{Gyr}\).}
  \label{fig:plateau_precoce}
\end{figure}

Pour \(T\ll T_{p}\), \(\dot P\approx0\) et donc \(P(T)\approx1\),
alors que pour \(T\gtrsim T_{p}\), l’intégration de \(\dot P\) relance la croissance selon la loi logistique.
Dans la limite \(T\to+\infty\), \(\alpha\to1\) et \(P(T)\sim T\), assurant \(I_{1}\to1\) conformément à la relativité générale.

\subsection{Motivations fondamentales et portée multidisciplinaire}

\paragraph{Limites du modèle \(\Lambda\)CDM}
Le modèle standard \(\Lambda\)CDM, malgré son succès, présente plusieurs tensions :
\begin{itemize}
  \item Formation précoce de galaxies massives (\(T\sim0.3\) Gyr) détectées par le JWST (JADES ; CEERS) \cite{JWST:JADES,JWST:CEERS}.
  \item Tension entre la mesure locale \(H_{0}^{\rm loc}\approx73\)\,\(\mathrm{km/s/Mpc}\) et la valeur issue du CMB \(H_{0}^{\rm CMB}\approx67\)\,\mathrm{km\,s^{-1}\,Mpc^{-1}} (écart ≃ 6 \%).
  \item Signatures potentielles sur le phasage des ondes gravitationnelles (LIGO/Virgo, LISA).
\end{itemize}
Le MCGT propose, via une modulation logistique de la composante temporelle—intégrant explicitement la dérivée \(\dot P(T)\)—et un couplage sombre modéré, de réduire ces écarts sans introduire d’éléments exotiques supplémentaires.

\subsection{Impact sur la tension de Hubble}

Grâce à l’intégration de \(\dot P(T)\) et au couplage sombre modéré\cite{Smith2024_CouplageSombre}, le MCGT induit, à la recombinaison, une légère correction de l’expansion précoce :
\[
  \frac{\delta H_{0}}{H_{0}}
  = \Bigl.\frac{\partial \ln P}{\partial \ln a}\Bigr|_{\mathrm{CMB}}
  \simeq
  \Bigl.\frac{P/a - 1}{P/a}\Bigr|_{z_{\mathrm{rec}}}
  \approx 9{,}2\times10^{-4},
\]
où \(a=(1+z)^{-1}\).

\begin{itemize}
  \item En \(\Lambda\mathrm{CDM}\) :
    \(H_{0}^{\rm CMB}\approx67\)\,\(\mathrm{km\,s^{-1}\,Mpc^{-1}}\),
    \(H_{0}^{\rm loc}\approx73\)\,\(\mathrm{km\,s^{-1}\,Mpc^{-1}}\),
    soit un décalage d’environ 6 \%.
  \item Sous MCGT, l’écart résiduel passe de 6 \% à 2.1 \% (grille v3),
    atténuant significativement la tension.
  \item L’introduction de couplages additionnels ou de degrés de liberté dynamiques
    dans \(\dot P(T)\) pourrait encore accroître \(\delta H_{0}\) et tendre vers
    une résolution complète de la tension.
\end{itemize}

Les valeurs chiffrées proviennent du fichier
\texttt{12-donnees/chapitre1/01\_H0\_tension\_MCGT.csv}.

\paragraph{Applications et domaines d’intérêt}
\begin{itemize}
  \item Cosmologie précoce : modulation de la croissance accélérée des structures
        explicative des objets détectés par le JWST.
  \item Ondes gravitationnelles : correction du décalage temporel et de phase
        dans l’analyse des signaux LIGO/Virgo et futurs signaux LISA.
  \item Couplage sombre : réconciliation partielle des mesures SNIa, BAO
        et \(H_{0}\) (voir Chapitre 8).
\end{itemize}

Les chapitres 2 à 4 détaillent la validation observationnelle,
tandis que le Chapitre 8 explore l’extension du couplage sombre.

\subsection{Principes structurants du MCGT}

\noindent Voir section « Cadre théorique et paramétrisation logistique » pour la définition de \(\alpha_{\mathrm{log}}(T)\), \(\alpha(T)\) et \(P(T)=T^{\alpha(T)}\).

\subsubsection{Extension \(f(R)\) du gravitationnel}

Le MCGT s’inscrit dans la classe des modifications \(f(R)\) de forme minimale
\[
  f(R) = R + \beta\,R^{n},
  \quad n>1,
\]
avec \(\beta\) et \(n\) ajustés par best-fit (Tab.~\ref{tab:FR_fit}) \cite{MCGT_FR_analysis}.
Les dérivées sont notées :
\[
  f_{R}\equiv\frac{\mathrm d f}{\mathrm d R},
  \quad
  f_{RR}\equiv\frac{\mathrm d^{2} f}{\mathrm d R^{2}}.
\]
La validité de ce modèle couvre la gamme de courbure
\(\lvert R\rvert\in[10^{-5},10^{4}]\,H_{0}^{2}\).
On définit l’invariant scalaire
\[
  m_{s}^{2}(R)
  = \frac{f_{R}(R) - R\,f_{RR}(R)}{3\,f_{RR}(R)}.
\]
Les conditions de stabilité linéaire sont alors :
\[
  f_{R}(R) > 0,\quad f_{RR}(R) > 0,\quad m_{s}^{2}(R) > 0,
\]
cette dernière assurant l’absence de tachyon et la stabilité du mode scalaire associé.
Les détails chiffrés et les tracés se trouvent au Chapitre 3 \cite{MCGT_FR_analysis}.

\subsubsection{Couplage sombre modéré}

Pour réconcilier les mesures de \(H_{0}\) et tenir compte des observations SNIa/BAO, on introduit un couplage entre matière sombre (\(\rho_{m}\)) et énergie sombre (\(\rho_{\phi}\)) :
\[
  \dot{\rho}_{m} + 3H\,\rho_{m} = -Q_{0}\,H_{0}\,\rho_{m},
\qquad
  \dot{\rho}_{\phi} + 3H\,(1+w_{\phi})\,\rho_{\phi} = +Q_{0}\,H_{0}\,\rho_{m},
\quad w_{\phi}=-1,
\]
où
\[
  Q_{0}\in[0.0,0.2]\quad\text{(adimensionné)},
  \quad
  H_{0}=67.4\,\mathrm{km\,s^{-1}\,Mpc^{-1}}\;(\text{Planck 2018}).
\]
Le meilleur ajustement donne
\[
  Q_{0}^{\mathrm{best}} = 0.04 \pm 0.02
  \quad(\text{voir Tab.~\ref{tab:Q0_fit}})\cite{Smith2024_CouplageSombre}.
\]
Les calculs détaillés, la structure des données et les résultats se trouvent au Chapitre 8.

\subsection{Renvois et perspectives}\label{sec:renvois}

Pour tout approfondissement\,\footnote{%
Les scripts et jeux de données associés sont référencés dans le
guide de chaque chapitre.} :
\begin{itemize}
  \item \textbf{Chapitre 2} : calibration chronologique, description des fichiers \texttt{.csv} et \texttt{.dat}.
  \item \textbf{Chapitre 3} : extension \(f(R)\) et analyse de stabilité linéaire.
  \item \textbf{Chapitre 4} : invariants adimensionnels \(I_{1},I_{2},I_{3}\) — définition, distribution et figures.
  \item \textbf{Chapitre 8} : couplage sombre modéré \((Q_{0})\) et impact sur les observables SNIa/BAO/\(H_{0}\).
\end{itemize}

\smallskip
\noindent\emph{Le volet conceptuel du Chapitre 1 s’achève à la section précédente. La partie qui suit présente uniquement les applications chiffrées et la calibration logistique.}
