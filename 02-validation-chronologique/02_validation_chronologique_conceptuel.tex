\section{Chapitre 2 – Validation chronologique (conceptuel)}

\subsection{Formule de validation}

La validation chronologique consiste à vérifier, pour neuf âges clés \(T_{i}\), que la fonction propre
\[
  P_{\rm calc}(T)
  = T^{\alpha(T)},
\]
issue de l’intégration complète de
\[
  \dot P(T)
  = \alpha(T)\,T^{\alpha(T)-1}
  + T^{\alpha(T)}\,\ln T\,\frac{\mathrm d\alpha}{\mathrm dT},
\]
Pour la forme analytique de $\alpha(T)$ et de $\mathrm d\alpha/\mathrm dT$, voir la section « Cadre théorique et paramétrisation logistique » (Chapitre 1).  
coïncide avec la valeur de référence \(P_{\rm ref}(T_i)\) obtenue par calibration fine.

Pour la forme analytique de $\alpha(T)$ et de $\mathrm d\alpha/\mathrm dT$, voir la section « Cadre théorique et paramétrisation logistique » (Chapitre 1).

On définit la fonction de coût
\[
  \chi^{2}
  = \sum_{i=1}^{9}
    \bigl[P_{\rm calc}(T_{i}) - P_{\rm ref}(T_{i})\bigr]^{2},
\]
et on ajuste les paramètres 
\(\alpha_{0},\,T_{c},\,\Delta,\,T_{p}\)
pour minimiser \(\chi^{2}\).  

Les âges clés et leurs valeurs de référence sont listés dans   
\texttt{zz-data/chapter02/02\_chronologie\_resultats.csv}.

\begin{table}[htbp]
  \centering
  \begin{tabular}{cc}
    \toprule
    Âge \(T_i\) (Gyr) & \(P_{\rm ref}(T_i)\)    \\
    \midrule
    \(10^{-6}\)           & 1.000000    \\
    \(10^{-5}\)           & 1.000000    \\
    \(3.8 \times 10^{-4}\)& 0.999999    \\
    \(10^{-3}\)           & 0.999992    \\
    \(2\times10^{-2}\)    & 0.865000    \\
    \(0.90\)              & 0.900000    \\
    \(3.00\)              & 3.273300    \\
    \(5.00\)              & 5.000000    \\
    \(13.80\)             & 13.800000   \\
    \bottomrule
  \end{tabular}
  \caption{Neuf âges clés et valeurs de référence pour la calibration chronologique.}
  \label{tab:jalons_chap2}
\end{table}
Pour visualiser la concordance entre la modélisation et les points de référence, voir Fig.~\ref{fig:p_vs_t_calibration} pour la courbe \(P(T)\) vs \(T\).

\subsection{Critères d’acceptation}

Les paramètres sont validés si l’écart relatif maximal  
\[
  \max_{i}\,\varepsilon(T_{i})
  = \max_{i}\,\frac{\lvert P_{\rm calc}(T_{i}) - P_{\rm ref}(T_{i})\rvert}{P_{\rm ref}(T_{i})}
  < 1\%.
\]
Ce critère sera démontré numériquement dans la section « Résultats numériques et cohérence ».

\subsection{Conclusion conceptuelle}

La fonction \(P(T)\), combinée à la comparaison systématique sur neuf âges clés, constitue la validation chronologique de base.  
Pour passer des définitions à la mise en œuvre concrète — c’est-à-dire pour consulter la structure exacte des fichiers de données, les extraits de CSV, les légendes des figures et les résultats numériques détaillés — se reporter à :

\begin{center}
  \texttt{02-validation-chronologique/02\_validation\_chronologique\_details.tex}
\end{center}

\noindent\emph{Script : \texttt{zz-scripts/chapter02/integration\_validation\_chronologie.py}, durée d’exécution \(\approx2\) min sur un laptop standard.}

\noindent\emph{Fin du volet conceptuel du Chapitre 2. La partie opérationnelle détaillée commence ci-dessous.}
