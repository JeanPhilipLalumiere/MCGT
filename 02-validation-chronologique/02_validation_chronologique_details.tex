\subsection{Fichiers de données}

La liste détaillée des fichiers de données (formats CSV et DAT) et leur descriptif complet se trouve dans le guide d’utilisation : \texttt{CHAPITRE2\_GUIDE.txt}.

\subsection{Script et mise en œuvre}

Les dépendances Python nécessaires sont listées dans \texttt{zz-scripts/chapter02/requirements.txt}.
Pour les installer, exécuter :
\begin{verbatim}
pip install -r zz-scripts/chapter02/requirements.txt
\end{verbatim}

Les données brutes et les figures sont générées par quatre scripts Python dans le dossier \texttt{zz-scripts/chapter02/} :

\begin{verbatim}
# 1. Intégration et calcul des tables
python zz-scripts/chapter02/integration_validation_chronologie.py
# Durée : ≈2 minutes

# 2. Tracés des figures de validation
python zz-scripts/chapter02/trace_fig01.py   # Fig 01 – P(T) vs T
python zz-scripts/chapter02/trace_fig02.py   # Fig 02 – Calibration 9 jalons
python zz-scripts/chapter02/trace_fig03.py   # Fig 03 – Écart relatif
python zz-scripts/chapter02/trace_fig04_schema_calibration.py
                                            # Fig 04 – Schéma de la chaîne
# Durée : quelques secondes par script
\end{verbatim}

\noindent\textbf{Note :} le script \texttt{integration\_validation\_chronologie.py} normalise la fonction propre en fixant
\(\displaystyle P\bigl(T_{0}=10^{-6}\,\mathrm{Gyr}\bigr)=1\).

\begin{mdframed}
  \paragraph{Encadré — Exécution et résultats}
  Après exécution des quatre scripts, le maximum d’écart relatif atteint est
  \[
    \max_{i}\,\varepsilon(T_i)\approx0{,}083\%\;<1\%.
  \]
\end{mdframed}

Les fichiers produits se trouvent dans :
\begin{itemize}
  \item \texttt{zz-data/chapter02/} pour les tables \texttt{.csv} et \texttt{.dat}.
  \item \texttt{zz-figures/chapter02/} pour les figures \texttt{.png}.
\end{itemize}

\subsection{Fig.~04 – Schéma de la chaîne de calibration}

Avant de présenter les résultats numériques, illustrons le fonctionnement général par le schéma de la chaîne de calibration :

\begin{figure}[htbp]
  \centering
  \includegraphics{zz-figures/chapter02/fig_04_schema_chaine_calibration.png}
  \caption{Chaîne de calibration chronologique. Ce diagramme, généré par \texttt{zz-scripts/chapter02/trace\_fig04\_schema\_calibration.py}, décrit :
    (i) lecture des neuf jalons \((T_i,P_{\rm ref})\),
    (ii) génération du tableau dense \(P(T)\) (\texttt{02\_P\_lin\_vs\_T.dat}),
    (iii) optimisation de \(\chi^2\), et
    (iv) sélection du meilleur jeu de paramètres.}
  \label{fig:schema_chaine_calibration}
\end{figure}

\subsection{Figures de validation}

Les résultats de la validation chronologique sont illustrés par trois figures, générées dans \texttt{zz-figures/chapter02/} :

\begin{itemize}
  \item Figure \ref{fig:p_vs_t_calibration_ch2} :
    trace dense de \(P_{\rm calc}(T)\) vs \(T\) (échelle log–log) avec superposition des neuf points de référence.
    Fichier : \texttt{fig\_01\_P\_vs\_T.png}.

  \item Figure \ref{fig:calibration_9points} :
    diagramme de calibration (échelle log–log) : scatter des neuf jalons avec barres d’erreur \(\pm1\%\) et courbe modèle.
    Fichier : \texttt{fig\_02\_calibration\_9points.png}.

  \item Figure \ref{fig:ecart_relatif_chronologie_ch2} :
    écart relatif \(\varepsilon(T_i)\) vs \(T_i\) en échelle log–log, confirmant tous les écarts \(<1\%\).
    Fichier : \texttt{fig\_03\_ecart\_relatif.png}.
\end{itemize}

\subsection{Résultats numériques et cohérence}

Les écarts relatifs
\[
  \varepsilon(T_{i})
  = \frac{\bigl|P_{\rm calc}(T_{i}) - P_{\rm ref}(T_{i})\bigr|}
         {P_{\rm ref}(T_{i})}\times100\%
\]
restent inférieurs à 1 % aux neuf âges clés, confirmant la réussite de la calibration.
Les valeurs numériques complètes sont disponibles dans
\texttt{zz-data/chapter02/02\_ecart\_relatif\_chronologie.csv}.

\subsubsection*{Discussion rapide des erreurs systématiques}

Les principales sources d’incertitude dans cette validation chronologique sont :

\begin{itemize}
  \item \textbf{Intégration trapézoïdale sur grille log :}
    le calcul de \(P(T)\) par intégration trapézoïdale est effectué sur une grille logarithmique de 800 points entre \(10^{-3}\) et \(14\) Gyr.
    L’erreur numérique est de l’ordre de \(\mathcal{O}\bigl((\Delta \ln T)^2\bigr)\approx10^{-5}\), contrôlée par la densité de la grille.

  \item \textbf{Précision des âges clés :}
    un décalage de \(\pm1\%\) sur un jalon \(T_i\) se traduit par une variation de \(\varepsilon(T_i)\) d’environ \(\pm0{,}1\%\).

  \item \textbf{Paramètres de plateau précoce et transition :}
    le choix des paramètres \(T_{p}\) et \(\Delta\) module la forme de \(\dot P(T)\) autour de la transition logistique.
    Leur estimation peut biaiser localement \(P(T)\), avec un effet global généralement inférieur à \(0{,}2\%\) sur \(\varepsilon_{\max}\).

  \item \textbf{Erreur d’interpolation :}
    lors de l’évaluation de \(P_{\rm calc}(T_i)\) sur les jalons, l’interpolation linéaire en \(\log T\) introduit une imprécision de l’ordre de \(10^{-4}\).

  \item \textbf{Artefact numérique et seuil minimal :}
    pour \(T<10^{-3}\) Gyr, on fixe \(T_{\min}=10^{-3}\) Gyr comme borne inférieure pour éviter les instabilités en double précision.
\end{itemize}

\subsection{Conclusion détaillée}

\noindent\textbf{Rappel :} le critère d’acceptation impose
\(\max_{i}\,\varepsilon(T_i)<1\%\).

La comparaison entre \(P_{\rm calc}(T)\) et \(P_{\rm ref}(T)\) sur l’ensemble des neuf âges clés valide la fonction propre avec une précision meilleure que \(1\%\), depuis l’ère planckienne (\(T_{0}=10^{-6}\,\mathrm{Gyr}\)) jusqu’à \(T=13{,}8\) Gyr.
Cette validation chronologique, à la fois robuste et précise, constitue la base solide pour les analyses suivantes : BBN (Chapitre 5), CMB (Chapitre 6) et perturbations scalaires (Chapitre 7).

\subsubsection*{Transition vers le Chapitre 3}

Les paramètres logistiques optimisés et validés ici seront désormais réutilisés pour étudier l’extension \(f(R)\) du modèle et analyser la stabilité linéaire des solutions, telles que développées dans le Chapitre 3.

\subsection{Glossaire}
\begin{itemize}
  \item $T_i$ : neuf âges clés de l’Univers (Gyr), définis dans \texttt{zz-data/chapter02/02\_chronologie\_resultats.csv}.
  \item $P_{\rm calc}(T)$ : fonction propre du temps calculée par intégration de $\dot P(T)$,
    \[
      P_{\rm calc}(T)=T^{\alpha(T)}.
    \]
  \item $P_{\rm ref}(T_i)$ : valeur de référence de la fonction propre aux âges clés, tirée de \texttt{zz-data/chapter02/02\_chronologie\_resultats.csv}.
  \item $\chi^2$ : fonction de coût pour la calibration,
    \[
      \chi^2 = \sum_{i=1}^{9}\bigl[P_{\rm calc}(T_i)-P_{\rm ref}(T_i)\bigr]^{2}.
    \]
  \item $\varepsilon(T_i)$ : écart relatif en pourcentage,
    \[
      \varepsilon(T_i)
      = \frac{\lvert P_{\rm calc}(T_i)-P_{\rm ref}(T_i)\rvert}
             {P_{\rm ref}(T_i)}\times100\%.
    \]
  \item \texttt{zz-data/chapter02/02\_chronologie\_resultats.csv} : neuf paires $(T_i,P_{\rm ref}(T_i))$.
  \item \texttt{zz-data/chapter02/02\_P\_lin\_vs\_T.dat} : table dense \((T,\,P_{\rm calc}(T))\) calculée par intégration trapézoïdale sur une grille logarithmique de 800 points entre \(T_{\min}=10^{-3}\)\,Gyr et \(14\)\,Gyr, avec condition initiale normalisée \(P(T_{0}=10^{-6}\,\mathrm{Gyr})=1\).
  \item \texttt{zz-data/chapter02/02\_deriveeP\_en\_fonction\_de_T.dat} : table dense $(T,\dot P(T))$ sur la même grille.
  \item \texttt{zz-data/chapter02/02\_ecart\_relatif\_chronologie.csv} : écarts relatifs $\varepsilon(T_i)$ aux neuf jalons.
\end{itemize}

\bigskip
\noindent\emph{Fin de la partie détaillée, Chapitre 2.}
