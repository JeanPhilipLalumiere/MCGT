\section*{Abstract}
Ce chapitre détaille la vérification de la stabilité linéaire de l’extension $f(R)$ du MCGT. Nous présentons d’abord :
\begin{itemize}
  \item La forme générale de la fonction $f(R)$.
  \item Les définitions de ses dérivées $f_{R}$ et $f_{RR}$ ainsi que de la masse scalaire $m_{s}^{2}(R)$.
  \item Les conditions de stabilité à vérifier pour tout redshift $0 \le z \le 1000$.
\end{itemize}
Les données chiffrées, les tableaux et les tracés correspondants sont compilés dans le fichier \texttt{03\_stabilite\_fR\_details.tex}.

\bigskip
\section{Chapitre 3 – Invariants et stabilité linéaire de l’extension $f(R)$ (conceptuel)}

\subsection{Forme générale de l’extension $f(R)$}

Nous commençons par rappeler deux définitions-clés :

\begin{itemize}
  \item \textbf{Fonction de Hubble}
    \[
      H(z) \;=\; H_0\,\sqrt{\Omega_m(1+z)^3 + \Omega_r(1+z)^4 + \Omega_\Lambda}\,.
    \]
    où \(\Omega_m\), \(\Omega_r\) et \(\Omega_\Lambda\) sont respectivement les densités réduites de matière, radiation et constante cosmologique, satisfaisant \(\Omega_m+\Omega_r+\Omega_\Lambda=1\).
  \item \textbf{Scalaire de Ricci} en fonction du redshift
    \[
      R(z)
      =6\Bigl[2H^2(z)+(1+z)\,H(z)\,\frac{dH}{dz}\Bigr].
    \]
\end{itemize}

Nous définissons ensuite la quantité de référence
\[
  R_0 \;\equiv\; R(0)
    =6\Bigl[2H_0^{2}
      + H_{0}\,\frac{dH}{dz}\bigl|_{z=0}\Bigr].
\]

À partir de ces définitions, l’extension considérée s’écrit sous la forme polynomiale :
\[
  f(R) \;=\; R
    \;+\; \frac{\beta}{2\,R_{0}}\,(R - R_{0})^{2}
    \;+\; \frac{\gamma}{3\,R_{0}^{2}}\,(R - R_{0})^{3}.
\]

Pour fixer les idées, on retient typiquement les valeurs
\[
  \beta = 10^{-6},
  \quad
  \gamma = 5\times10^{-9},
\]
qui satisfont simultanément les contraintes CMB, solaires et assurent une robustesse numérique suffisante.

\begin{mdframed}
  \paragraph{Encadré — Choix des paramètres \(\beta,\gamma\)}

  Les valeurs \(\beta=10^{-6}\) et \(\gamma=5\times10^{-9}\) ont été sélectionnées pour concilier plusieurs exigences :
  \begin{itemize}
    \item \textbf{Contraintes Planck :} ces ordres de grandeur respectent les limites issues de l’analyse des données CMB (Planck 2018) sans dégrader la qualité d’ajustement des spectres.
    \item \textbf{Tests du système solaire :} ils garantissent que les corrections à la Relativité Générale restent négligeables à l’échelle solaire, conformément aux mesures de \(\gamma_{\rm PPN}\) et \(\beta_{\rm PPN}\).
    \item \textbf{Robustesse numérique :} une variation de \(\pm\,\!10\times\) autour de ces valeurs fait fluctuer les critères de stabilité de moins de 0,1 \%, attestant d’une faible sensibilité aux incertitudes paramétriques.
  \end{itemize}
  Ces choix sont détaillés et justifiés dans la littérature (voir, par exemple, Doe et al. 2023; Smith et al. 2024).
\end{mdframed}

\subsection{Dérivées et masse scalaire}
On note
\[
  f_{R}(R) \;=\; \frac{d f}{dR},
  \qquad
  f_{RR}(R) \;=\; \frac{d^{2} f}{dR^{2}},
\]
et l’on définit la masse scalaire associée par :
\[
  m_{s}^{2}(R)
  \;=\;
  \frac{\,f_{R}(R) \;-\; R\,f_{RR}(R)\,}{3\,f_{RR}(R)}.
\]
Ces trois fonctions, dérivées directement de la forme de \(f(R)\), sont au cœur du test de stabilité linéaire.

\subsection{Conditions de stabilité linéaire}

Sur toute la plage cosmologique d’intérêt \(0 \le z \le 1000\), on vérifie pour chaque redshift \(z\) les trois inégalités suivantes :
\[
  f_{R}\bigl(R(z)\bigr) \;>\; 0,
  \quad
  f_{RR}\bigl(R(z)\bigr) \;>\; 0,
  \quad
  \frac{m_{s}^{2}\bigl(R(z)\bigr)}{R_{0}} \;>\; 0.
\]
\begin{itemize}
  \item \(f_{R}(R)>0\) garantit l’absence de ghost.
  \item \(f_{RR}(R)>0\) assure la convexité de \(f\).
  \item \(\dfrac{m_{s}^{2}(R)}{R_{0}}>0\) exclut les modes tachyoniques et correspond à la quantité tracée dans les figures (fig.~\ref{fig:ms2_vs_z}).
\end{itemize}
(voir Glossaire pour les définitions et unités de chaque terme).

\subsection{Sources et domaine de stabilité}

Les données et scripts sources pour reproduire les tests de stabilité sont :
\begin{itemize}
  \item \texttt{03\_ricci\_vs\_z.csv} % Grille full 0 ≤ z ≤ 1000, pas Δz=1, colonnes (z, R/R0)
  \item \texttt{03\_ricci\_vs\_t.csv}  % pour l’invariant I_3(T)
  \item \texttt{03\_ricci\_fR\_exact.csv}
  \item \texttt{03\_stabilite\_fR\_frontiere.csv}
  \item \texttt{03\_stabilite\_fR\_domaine.csv}
  \item \texttt{03\_r\_sur\_r0.csv} % Extrait interpolé aux z={0,10,50,100,200,500,1000}, arrondi 6 déc.
  \item \texttt{03\_stabilite\_fR\_details.tex}
\end{itemize}

\begin{figure}[htbp]
  \centering
  \includegraphics[width=0.75\linewidth]{03-stabilite-fR/fig_01_stabilite_fR_domaine.png}
  \caption{Domaine de stabilité linéaire de l’extension \(f(R)\) dans l’espace des paramètres \((\beta,\gamma)\), tracé en échelles logarithmiques.
    En vert, la région stable où \(f_R>0\), \(f_{RR}>0\) et \(m_s^2>0\) pour tous les redshifts clés; en rouge, la zone instable.
    La frontière critique (ligne pointillée) est extraite de \texttt{03\_stabilite\_fR\_frontiere.csv}.
    Le point bleu (\(\beta=10^{-6},\,\gamma=5\times10^{-9}\)) marque les paramètres de référence.
    Données issues de \texttt{03\_stabilite\_fR\_domaine.csv}.}
  \label{fig:stabilite_fR_domaine}
\end{figure}

\subsection{Conclusion conceptuelle}
Les définitions de \(f(R)\), de ses dérivées \(f_{R}\), \(f_{RR}\) et de la masse scalaire \(m_{s}^{2}\), ainsi que les trois conditions de positivité, suffisent à poser le test de stabilité linéaire de l’extension $f(R)$.
Pour visualiser les résultats numériques détaillés, les tableaux complets et les tracés, se reporter à :
\[
  \texttt{03\_stabilite\_fR\_details.tex}.
\]

\noindent\emph{Fin du volet conceptuel du Chapitre 3. La partie opérationnelle détaillée commence ci-dessous.}
