\section*{Abstract}
Ce fichier présente le cadre conceptuel des invariants adimensionnels du MCGT. Il comprend :
\begin{itemize}
  \item les définitions formelles de $I_{1}(T)$, $I_{2}(T)$ et $I_{3}(T)$ ;
  \item leurs ordes de grandeur typiques ;
  \item des interprétations qualitatives, complétées par un tableau synthétique et un schéma conceptuel.
\end{itemize}
Se reporter à la partie « détails » pour les données numériques et les visualisations (\texttt{04\_invariants\_adimensionnels\_details.tex}).

\bigskip
\section{Chapitre 4 – Invariants adimensionnels du MCGT (conceptuel)}

\subsection{Définitions formelles des invariants}
Dans le MCGT, on introduit trois invariants adimensionnels en fonction de l’âge cosmique $T$ :
\[
  I_{1}(T) \;=\; \frac{P(T)}{T},
  \qquad
  I_{2}(T) \;=\; \kappa\,T^{2},
  \qquad
  I_{3}(T) \;=\; F\bigl(R(T)\bigr)\;-\;1,
\]
avec :
\begin{itemize}
  \item $P(T)$ : la fonction propre du temps (définie au Chapitre 1),
  \item $\kappa$ : constante adimensionnelle, de l’ordre de $10^{-35}$ (valeur précise donnée dans le fichier de données),
  \item $F(R) = f_{R}(R)$ : dérivée première de l’extension $f(R)$ (voir Chapitre 3),
  \item $R(T)$ : courbure scalaire en fonction de $T$ (voir Chapitre 3).
\end{itemize}

\subsection{Ordres de grandeur et interprétations qualitatives}
Les plages typiques observées pour chaque invariant sont :
\[
  I_{1}(T)\;\sim\;[10^{-3},\,2],
  \qquad
  I_{2}(T)\;\lesssim\;10^{-35},
  \qquad
  I_{3}(T)\;\lesssim\;10^{-6}.
\]
\begin{itemize}
  \item \textbf{$I_{1}(T) = P(T)/T$} : mesure le rapport du temps propre au temps cosmique,
    \begin{itemize}
      \item lorsque $I_{1}\approx 2$ (autour de $T\sim0{,}3$ Gyr), la croissance des fluctuations de densité est maximisée,
      \item pour $T\to 14$ Gyr, $I_{1}$ se stabilise proche de 1, ramenant MCGT vers $\Lambda$CDM.
    \end{itemize}
  \item \textbf{$I_{2}(T) = \kappa\,T^{2}$} : reste extrêmement faible ($\lesssim10^{-35}$) sur toute la durée cosmique,
    \begin{itemize}
      \item $\kappa$ a été déterminé de manière à garantir une influence négligeable sur l’évolution globale,
      \item cet invariant permet de diagnostiquer la contribution purement adimensionnelle du paramétrage temporel.
    \end{itemize}
  \item \textbf{$I_{3}(T) = F(R) - 1$} : suit la déviation de la dérivée $f_{R}(R)$,
    \begin{itemize}
      \item $I_{3}\lesssim10^{-6}$ indique que $f(R)$ reste très proche de la forme linéaire $R$,
      \item cette petite déviation reflète l’impact modifié du gravitationnel sur la composante temporelle.
    \end{itemize}
\end{itemize}

\subsection{Tableau synthétique des invariants}

Ci-dessous, un résumé compact des trois invariants adimensionnels définis précédemment, assorti de leurs plages typiques et de leur interprétation physique.

\begin{table}[htbp]
  \centering
  \caption{Tableau synthétique des invariants adimensionnels du MCGT}
  \label{tab:invariants_synth}
  \begin{tabular}{l l l p{0.45\textwidth}}
    \toprule
    Invariant & Formule & Plage typique & Interprétation clé \\
    \midrule
    \(I_{1}(T)\) & \(\displaystyle \frac{P(T)}{T}\)
                  & \([10^{-3},\,2]\)
                  & Mesure le rapport du temps propre au temps cosmique. \\
    \(I_{2}(T)\) & \(\displaystyle \kappa\,T^{2}\)
                  & \(\lesssim 10^{-35}\)
                  & Quantifie la contribution purement adimensionnelle du paramétrage temporel. \\
    \(I_{3}(T)\) & \(\displaystyle F(R(T)) - 1\)
                  & \(\lesssim 10^{-6}\)
                  & Exprime la déviation de la dérivée de l’extension gravitationnelle. \\
    \bottomrule
  \end{tabular}
\end{table}

Ce tableau offre une vue d’ensemble rapide, facilitant la comparaison des invariants et leur rôle respectif dans le cadre du MCGT.

\subsection{Schéma conceptuel de l’évolution des invariants}

Pour visualiser en un coup d’œil la dynamique relative des trois invariants adimensionnels, nous présentons ci-dessous un schéma conceptuel tracé en fonction de \(\log T\).

\begin{figure}[htbp]
  \centering
  \includegraphics[width=0.85\linewidth]{04-invariants-adimensionnels/fig_01_schema_invariants_adimensionnels.png}
  \caption{Schéma conceptuel de l’évolution des invariants \(I_{1},I_{2},I_{3}\) en fonction de \(\log T\).
    Les séparateurs verticaux délimitent trois phases clés de l’histoire cosmique,
    et les lignes pointillées oranges indiquent respectivement les niveaux de référence
    \(I_{2}\approx10^{-35}\) et \(I_{3}\approx10^{-6}\).}
  \label{fig:04_schema_invariants}
\end{figure}

Le diagramme comporte :

\begin{itemize}
  \item Trois \textbf{phases temporelles} mises en évidence par des lignes verticales :
    \begin{itemize}
      \item \textbf{Précocité – Plateau bas}
        : \(I_{1}\ll1\), phase embryonnaire où la croissance des structures est naissante.
      \item \textbf{Transition logistique – Pic de \(I_{1}\)}
        : augmentation rapide de \(I_{1}\), signe de l’accélération des fluctuations.
      \item \textbf{Régime tardif – Retour vers 1}
        : \(I_{1}\to1\), convergence progressive du MCGT vers le comportement \(\Lambda\)CDM.
    \end{itemize}
  \item Deux \textbf{lignes horizontales pointillées} (en orange) :
    \begin{itemize}
      \item \(I_{2}\approx10^{-35}\)  -- contribution adimensionnelle quasi-constante.
      \item \(I_{3}\approx10^{-6}\)   -- déviation minimale de la dérivée \(f_{R}(R)\).
    \end{itemize}
  \item Une flèche noire épaisse souligne l’avancée de \(\log T\)
        de la gauche (univers primordial) vers la droite (univers tardif).
\end{itemize}

Ce schéma, complété par le « Tableau synthétique des invariants », offre une vision pédagogique
de la manière dont chacun des trois invariants se comporte au cours de l’évolution cosmique.

\subsection{Conclusion conceptuelle}
Les définitions formelles et les ordres de grandeur présentés ici suffisent à comprendre le rôle de chacun des trois invariants dans le diagnostic MCGT.
Pour les détails d’implémentation (scripts, données et figures), se reporter à \texttt{04\_invariants\_adimensionnels\_details.tex}.

\subsection{Perspectives et applications futures}

À l’issue de la partie conceptuelle, ces invariants adimensionnels ouvrent plusieurs pistes d’exploitation :

\begin{quote}
« Ces trois grandeurs adimensionnelles pourront servir de benchmarks pour comparer d’autres modèles modifiés ; en particulier, l’évolution de \(I_{1}(T)\) a un impact direct sur la croissance des structures, tandis que \(I_{3}(T)\) offre une jauge de la robustesse des extensions \(f(R)\). »
\end{quote}

\noindent\emph{Fin du volet conceptuel du Chapitre 4. La partie opérationnelle détaillée commence ci-dessous.}
