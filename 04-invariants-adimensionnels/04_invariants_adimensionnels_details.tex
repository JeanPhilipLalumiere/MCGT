\subsection{Fichier de données principal : \texttt{04\_invariants\_adimensionnels\_complet.csv}}

\vspace{0.5em}
\noindent
Le fichier \texttt{04\_invariants\_adimensionnels\_complet.csv} contient 1400 points correspondant à une grille régulière en $T\in[10^{-3},\,14]$ Gyr (pas $\Delta T = 0{,}01$ Gyr) :
\[
  T,\;I_{1}(T),\;I_{2}(T),\;I_{3}(T).
\]
Les valeurs ont été générées par le script \texttt{13\_calculer\_invariants\_dimensionnels.py}.

\medskip
\noindent
\textbf{Exemple de structure (quelques lignes illustratives) :}
\begin{center}
\begin{tabular}{S[table-format=2.3] S[table-format=1.6] S[table-format=1.6] S[table-format=1.6]}
\toprule
{\(T\) (Gyr)} & {\(I_{1}(T)\)} & {\(I_{2}(T)\)} & {\(I_{3}(T)\)} \\
\midrule
0.001  & 0.001234 & 1.000e-35 & 3.456e-07 \\
0.005  & 0.005678 & 2.500e-34 & 2.987e-07 \\
0.010  & 0.010123 & 1.000e-33 & 2.500e-07 \\
\vdots & \vdots   & \vdots    & \vdots    \\
13.800 & 1.000987 & 1.900e-34 & 1.234e-06 \\
\bottomrule
\end{tabular}
\end{center}

\bigskip
\subsection{Histogramme en échelle logarithmique des invariants}

\begin{figure}[htbp]
  \centering
  \includegraphics[width=0.75\linewidth]{04-invariants-adimensionnels/fig_02_histogramme_invariants_adimensionnels.png}
  \caption{Histogramme des distributions de \(\log_{10}I_{2}\) (jaune) et \(\log_{10}I_{3}\) (orange), normalisé sur l’ensemble de \(T\in[10^{-3},14]\) Gyr.}
  \label{fig:hist_invariants}
\end{figure}

\subsubsection*{Description du fichier}
\begin{itemize}
  \item \textbf{Nom :} \texttt{fig\_02\_histogramme\_invariants\_adimensionnels.png}
  \item \textbf{Type :} image PNG (800 × 500 px)
  \item \textbf{Génération :} script \texttt{13\_tracer\_invariants\_adimensionnels.py}, à partir de \texttt{04\_invariants\_adimensionnels\_complet.csv}
  \item \textbf{Contenu :} histogramme normalisé des log–valeurs des invariants
    \(
      \log_{10}I_{2}
    \)
    et
    \(
      \log_{10}I_{3}
    \)
    \begin{itemize}
      \item \textbf{Axe horizontal :} \(\log_{10}(\text{valeur de l’invariant})\).
      \item \textbf{Axe vertical :} fréquence normalisée.
      \item \textbf{Couleurs :}
        \begin{itemize}
          \item Jaune pour \(\log_{10}I_{2}\),
          \item Orange pour \(\log_{10}I_{3}\).
        \end{itemize}
    \end{itemize}
\end{itemize}

\subsubsection*{Interprétation quantitative}
\begin{itemize}
  \item \(\log_{10}I_{2}\) (jaune) : pic étroit autour de \(\approx -35\), reflétant la très faible amplitude et la quasi-constance de \(I_{2}\).
  \item \(\log_{10}I_{3}\) (orange) : distribution centrée vers \(\approx -6\), indiquant que la déviation gravitationnelle reste faible mais plus variable que \(I_{2}\).
\end{itemize}

\bigskip
\subsection{Évolution log–log des invariants}

\begin{figure}[htbp]
  \centering
  \includegraphics[width=0.85\linewidth]{04-invariants-adimensionnels/fig_03_i1_i2_i3_vs_t.png}
  \caption{Invariants adimensionnels \(I_{1},I_{2},I_{3}\) en fonction de l’âge cosmique \(T\) (Gyr) sur échelle log–log.
    — Jaune : \(I_{1}=P(T)/T\)
    — Orange : \(I_{2}=\kappa\,T^{2}\)
    — Rouge : \(I_{3}=F(R(T))-1\)}
  \label{fig:invariants_vs_t}
\end{figure}

\subsubsection*{Description du fichier}
\begin{itemize}
  \item \textbf{Nom :} \texttt{fig\_03\_i1\_i2\_i3\_vs\_t.png}
  \item \textbf{Type :} PNG (800×500 px)
  \item \textbf{Génération :} script \texttt{13\_tracer\_invariants\_adimensionnels.py}, à partir de \texttt{04\_invariants\_adimensionnels\_complet.csv}.
  \item \textbf{Axes :}
    \begin{itemize}
      \item Abscisse : \(T\) en Gyr (logarithmique).
      \item Ordonnée : valeur adimensionnelle de l’invariant (logarithmique).
    \end{itemize}
  \item \textbf{Courbes :}
    \begin{itemize}
      \item Jaune — \(I_{1}(T)=P(T)/T\)
      \item Orange — \(I_{2}(T)=\kappa\,T^{2}\)
      \item Rouge — \(I_{3}(T)=F\bigl(R(T)\bigr)-1\)
    \end{itemize}
\end{itemize}

\subsubsection*{Analyse des tendances}
\begin{itemize}
  \item \textbf{\(I_{1}\) (jaune)} :
    décroît de \(\sim10^{2}\) à \(\sim10^{1}\) avant de se stabiliser, traduisant la convergence tardive vers le comportement \(\Lambda\)CDM.
  \item \textbf{\(I_{2}\) (orange)} :
    augmente progressivement de \(\sim10^{-42}\) à \(\sim10^{-34}\), conformément à la loi \(\kappa\,T^{2}\).
  \item \textbf{\(I_{3}\) (rouge)} :
    décroît fortement de l’ordre de \(10^{0}\) à \(10^{-10}\), reflétant la diminution rapide de la déviation gravitationnelle \(f_{R}(R)-1\).
  \item \textbf{Synthèse :}
    On observe une montée abrupte de \(I_{1}\) lors de la phase de transition logistique, suivie d’une stabilisation autour de 1, tandis que \(I_{2}\) et \(I_{3}\) varient sur plusieurs ordres de grandeur conformément à leurs définitions respectives.
\end{itemize}

\subsection*{Glossaire}

\begin{description}
  \item[$T$] Âge cosmique, mesuré en milliards d’années (Gyr).
  \item[$P(T)$] Fonction propre du temps, correspondant au temps mesuré par une horloge locale.
  \item[$I_{1}(T)$] Invariant adimensionnel $I_{1}(T)=P(T)/T$, mesurant le rapport du temps propre au temps cosmique.
  \item[$I_{2}(T)$] Invariant adimensionnel $I_{2}(T)=\kappa\,T^{2}$, quantifiant la contribution purement adimensionnelle du paramètre temporel.
  \item[$I_{3}(T)$] Invariant adimensionnel $I_{3}(T)=F(R(T))-1$, exprimant la déviation de la dérivée première de la fonction $f(R)$.
  \item[$\kappa$] Constante adimensionnelle de l’ordre de $10^{-35}$, choisie pour minimiser l’impact de $I_{2}$ sur l’évolution cosmique.
  \item[$F(R)$] Dérivée première de la fonction $f(R)$, utilisée pour modéliser les extensions de la gravitation.
  \item[$R(T)$] Courbure scalaire de l’espace-temps en fonction de l’âge cosmique $T$.
  \item[MCGT] Modèle de Courbe Gravitationnelle du Temps.
\end{description}

\bigskip
\noindent\emph{Fin de la partie détaillée, Chapitre 4.}
