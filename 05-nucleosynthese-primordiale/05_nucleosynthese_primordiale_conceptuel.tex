\section*{Abstract}
Cadre conceptuel de la nucléosynthèse primordiale (BBN) dans le MCGT : contexte et objectifs, définition de la fonction de coût \(\chi^2_{\rm BBN}(T)\) et workflow conceptuel.
Pour le détail opérationnel (fichiers CSV, scripts, figures), se reporter à \texttt{05\_nucleosynthese\_primordiale\_details.tex}.

\vspace{1em}
\section{Chapitre 5 – Nucléosynthèse Primordiale (BBN) dans le MCGT (conceptuel)}

\subsection{Contexte et objectifs}

La nucléosynthèse primordiale se déroule lorsque l’âge cosmique est compris entre
\[
  T \sim 10^{-5}\text{–}10^{-3}\;\mathrm{Gyr}
  \quad(\sim0{,}01\text{–}10\,\mathrm{s}).
\]

\begin{tcolorbox}[colback=gray!10,colframe=black,title=Fenêtre temporelle critique]
BBN : \(T\in[10^{-5},\,10^{-3}]\;\mathrm{Gyr}\)
(soit \(\sim0{,}01\)–\(10\) s).
\end{tcolorbox}

\begin{table}[htbp]
  \centering
  \caption{Abondances primordiales observées (2023)}
  \begin{tabular}{lcc}
    \toprule
    Nucléide & Valeur & Incertitude \\
    \midrule
    \(Y_{p}\) & 0,2449 & \(\pm0,0040\) \\
    \(D/H\)   & \(2,527\times10^{-5}\) & \(\pm0,030\times10^{-5}\) \\
    \bottomrule
  \end{tabular}
\end{table}

Dans le cadre MCGT, l’expansion est modifiée par la dérivée propre du temps \(\dot P(T)\) :
\[
  H_{\rm MCGT}(T)
  = \frac{H_{\Lambda\mathrm{CDM}}(T)}{\dot P(T)}
  \quad\Longleftrightarrow\quad
  \frac{H_{\rm MCGT}(T)}{H_{\Lambda\mathrm{CDM}}(T)}
  = \frac{1}{\dot P(T)}.
\]

L’objectif conceptuel est de vérifier, sur la fenêtre critique \(T\in[10^{-5},\,10^{-3}]\) Gyr :
\begin{enumerate}
  \item Calculer le ratio \(\displaystyle \frac{H_{\rm MCGT}(T)}{H_{\Lambda\mathrm{CDM}}(T)} = 1/\dot P(T)\) pour chaque \(T\).
  \item Coupler ce facteur d’expansion à AlterBBN pour obtenir les abondances modélisées
    \[
      Y_{p}^{\rm mod}(T)
      \quad\text{et}\quad
      (D/H)^{\rm mod}(T).
    \]
  \item Comparer ces abondances aux mesures référencées dans le tableau ci-dessus.
  \item Définir la fonction de coût \(\chi^{2}_{\rm BBN}(T)\) (voir la prochaine section) et vérifier qu’elle reste proche de 1 sur toute la plage.
\end{enumerate}

\subsection{Définition de la fonction de coût \(\chi^{2}_{\rm BBN}(T)\)}

La qualité d’accord entre les abondances modélisées et les observations de référence est quantifiée par la fonction de coût
\[
  \chi^{2}_{\rm BBN}(T)
  =
  \frac{\bigl[Y_{p}^{\rm mod}(T) - Y_{p}^{\rm obs}\bigr]^{2}}{\sigma_{Y_{p}}^{2}}
  +
  \frac{\bigl[(D/H)^{\rm mod}(T) - (D/H)^{\rm obs}\bigr]^{2}}{\sigma_{D/H}^{2}}.
\]
, les valeurs observées et l’intervalle critique étant définis dans la section « Contexte et objectifs ».

\subsection{Workflow conceptuel}

Le processus conceptuel se résume au schéma suivant :

\begin{figure}[htbp]
  \centering
  \includegraphics[width=0.85\linewidth]{05-nucleosynthese-primordiale/fig_01_schema_flux_bbn.png}
  \caption{Schéma de workflow : chargement de \(\dot P(T)\), préparation et exécution d’AlterBBN, puis calcul de \(\chi^2_{\rm BBN}(T)\).}
  \label{fig:flux_nucleosynthese}
\end{figure}

Le processus conceptuel de validation de la BBN en MCGT s’articule en quatre grandes étapes :
\begin{enumerate}
  \item Charger la table \(\dot P(T)\) depuis \texttt{02\_pdot\_tableau\_plateau\_fine.dat} (30 points, \(T\) converti en Gyr).
  \item Préparer le fichier \texttt{bbn\_input.txt} au format requis par AlterBBN.
  \item Exécuter AlterBBN pour obtenir les abondances modélisées \(Y_{p}^{\rm mod}(T)\) et \((D/H)^{\rm mod}(T)\).
  \item Calculer la fonction de coût \(\chi^{2}_{\rm BBN}(T)\) selon la définition (section « Définition de la fonction de coût \(\chi^{2}_{\rm BBN}(T)\) »).
\end{enumerate}

\subsection{Critères d’acceptation et conclusion conceptuelle}

\begin{itemize}
  \item \textbf{Concordance quantitative :}
    exiger \(\chi^{2}_{\rm BBN}(T)\lesssim10^{-2}\) sur l’ensemble de l’intervalle critique.
  \item \textbf{Concordance qualitative :}
    les abondances modélisées doivent tomber à l’intérieur des barres d’erreur des observations pour \(Y_{p}\) et \(D/H\).
  \item \textbf{Conclusion conceptuelle :}
    si ces critères sont remplis, le MCGT est compatible avec les abondances primordiales observées.
\end{itemize}

\subsection{Renvoi vers les détails opérationnels}

Pour tous les aspects techniques (format des fichiers CSV, scripts Python, exemples de sorties et figures), se reporter à :

\begin{center}
  \texttt{05\_nucleosynthese\_primordiale\_details.tex}
\end{center}

\noindent\emph{Fin du volet conceptuel du Chapitre 5. La partie opérationnelle détaillée commence ci-dessous.}
