\subsection{Source des données \(\dot P(T)\)}

La table \dot P(T) provient directement du Chapitre 2 :
\texttt{02_pdot_tableau_plateau_fine.dat}, contenant 30 points \{T_i,\dot P(T_i)\} couvrant
\[
  T\in[0{,}01,\,10]\;\mathrm{s},
\]
déjà exprimés en secondes.
Ce fichier constitue l’entrée de base pour la conversion vers le format AlterBBN (section « Format du fichier \texttt{bbn\_input.txt} »).

\subsubsection{Script \texttt{13\_executer\_alterbbn\_plateau.py}}

Ce script prend en entrée la table \texttt{02\_pdot\_tableau\_plateau\_fine.dat} (déjà exprimée en secondes) et génère le fichier \texttt{bbn\_input.txt} au format requis par AlterBBN.

\begin{itemize}
  \item \textbf{Usage} :
    \verb|python 13_executer_alterbbn_plateau.py|

  \item \textbf{Fonctionnement} :
    \begin{enumerate}
      \item Lit la table de points \(\{T_i,\dot P(T_i)\}\) au format deux colonnes, où \(T_i\) est déjà en secondes.
      \item Calcule le ratio
        \[
          r_i \;=\;\frac{1}{\dot P(T_i)}.
        \]
      \item Vérifie que
        \[
          r_i > 0
          \quad\text{et}\quad
          r_i \lesssim 1.5;
        \]
        en cas d’anomalie, le script stoppe avec un message d’erreur.
      \item Écrit le fichier \texttt{bbn\_input.txt} :
        \begin{itemize}
          \item Première ligne : commentaire
            \verb|# T(s) H_ratio|
          \item Lignes suivantes :
            \verb|<T_i> <r_i>|
            où <T_i> est la valeur de l’âge cosmique en secondes et <r_i> le ratio calculé.
        \end{itemize}
    \end{enumerate}

  \item \textbf{Dépendances} :
    Python 3.x, \texttt{numpy}.
\end{itemize}

\subsection{Format du fichier \texttt{bbn\_input.txt}}

Le fichier \texttt{bbn\_input.txt} doit respecter le format suivant :

\begin{itemize}
  \item Deux colonnes séparées par un espace :
    \[
      T(\mathrm{s}),\quad \frac{H_{\rm MCGT}(T)}{H_{\Lambda\mathrm{CDM}}(T)}.
    \]
  \item La première ligne est un commentaire débutant par \texttt{\#} :
    \begin{verbatim}
# T(s) H_ratio
    \end{verbatim}
  \item Les lignes suivantes contiennent un exemple de points :
    \begin{verbatim}
3.153600e+10 1.000000
6.307200e+10 0.980123
9.460800e+10 0.979456
...
3.153600e+13 0.970001
    \end{verbatim}
\end{itemize}

\subsection{Exécution d’AlterBBN}

\begin{itemize}
  \item \textbf{Préparation} :
    copier \texttt{bbn\_input.txt} dans le répertoire principal d’AlterBBN, sans modifier les paramètres par défaut (\(\Omega_{b}h^{2},\,N_{\nu}\), etc.).
  \item \textbf{Lancement} :
    \begin{verbatim}
alterbbn --input bbn_input.txt
    \end{verbatim}
  \item \textbf{Sortie} :
  AlterBBN génère le fichier \texttt{05\_nucleosynthese\_resultats\_exact.csv}, structuré en trois colonnes :
  \[
    T\,[\mathrm{Gyr}],\quad Y_{p}^{\rm mod}(T),\quad (D/H)^{\rm mod}(T).
  \]
\item \textbf{Exemple d’extrait} (30 lignes) :
  \begin{verbatim}
T(Gyr),Yp_mod,DH_mod
3.170979e-19,0.24500,2.530e-05
4.023849e-19,0.24500,2.530e-05
…
3.170979e-16,0.24500,2.530e-05
  \end{verbatim}
\end{itemize}

\subsection{Calcul de \(\chi^{2}_{\rm BBN}(T)\)}

Cette étape produit le fichier \texttt{05\_chi2\_nucleosynthese\_vs\_t.csv} contenant, pour chaque \(T\) de la grille, la valeur de la fonction de coût \(\chi^{2}_{\rm BBN}(T)\).

\subsubsection*{Script \texttt{13\_calculer\_chi2\_nucleosynthese.py}}

\begin{itemize}
  \item \textbf{Objectif} : charger \texttt{05\_nucleosynthese\_resultats\_exact.csv}, appliquer la formule définie en section « Définition de la fonction de coût \(\chi^{2}_{\rm BBN}(T)\) » et écrire \texttt{05\_chi2\_nucleosynthese\_vs\_t.csv}.
  \item \textbf{Usage} :
    \begin{verbatim}
python 13_calculer_chi2_nucleosynthese.py
    \end{verbatim}
  \item \textbf{Étapes essentielles} :
    \begin{enumerate}
      \item Lecture du CSV avec \texttt{pandas}.
      \item Calcul de \(\chi^{2}_{\rm BBN}(T)\) pour chaque ligne comme défini précédemment.
      \item Écriture du résultat sous la forme :
        \[
          T\,[\mathrm{Gyr}],\;\chi^{2}_{\rm BBN}(T).
        \]
    \end{enumerate}

  \item \textbf{Dépendances} : Python 3.x, \texttt{pandas}.
\end{itemize}

\subsection{Structure de \texttt{05\_chi2\_nucleosynthese\_vs\_t.csv}}
Ce fichier contient deux colonnes sans ligne de commentaire :
\[
  T\,[\mathrm{Gyr}],\quad \chi^{2}_{\rm BBN}(T).
\]
\noindent\textbf{Exemple d’extrait} (30 lignes) :
\begin{verbatim}
T(Gyr),chi2_BBN
3.170979e-19,0.0106
4.023849e-19,0.0106
…
3.170979e-16,0.0106
\end{verbatim}
\noindent\emph{Commentaires} :
\begin{itemize}
  \item Les valeurs de \chi^{2}_{\rm BBN} sont de l’ordre de 1,06\times10^{-2}, attestant d’une excellente concordance.
  \item L’ordre de tri est croissant en \(T\).
\end{itemize}

\subsection{Génération des figures comparatives}

\subsubsection*{Script \texttt{13\_tracer\_resultats\_nucleosynthese.py}}

\begin{itemize}
  \item \textbf{Objectif} : lire les fichiers \texttt{05\_nucleosynthese\_resultats\_exact.csv} et \texttt{05\_chi2\_nucleosynthese\_vs\_t.csv} pour produire trois graphiques :
    \begin{itemize}
      \item \texttt{fig\_02\_dh\_modele\_vs\_obs.png} : comparaison D/H observé vs modélisé,
      \item \texttt{fig\_03\_yp\_modele\_vs\_obs.png} : comparaison \(Y_{p}\) observé vs modélisé,
      \item \texttt{fig\_04\_chi2\_nucleosynthese\_vs\_t.png} : \(\chi^{2}_{\rm BBN}(T)\) en fonction de \(T\).
    \end{itemize}

  \item \textbf{Étapes principales} :
    \begin{enumerate}
      \item Charger les deux CSV avec \texttt{pandas}.
      \item Tracer un diagramme en barres pour D/H (obs & mod) et pour \(Y_{p}\) (obs & mod).
      \item Tracer en échelle semi-logarithmique \chi^{2}_{\rm BBN}(T) vs T, avec ligne de référence \chi^{2}=0{,}01.
      \item Sauvegarder chaque figure au format PNG.
    \end{enumerate}

  \item \textbf{Invocation} :
    \begin{verbatim}
python 13_tracer_resultats_nucleosynthese.py
    \end{verbatim}

  \item \textbf{Dépendances} : Python 3.x, \texttt{pandas}, \texttt{matplotlib}.
\end{itemize}

\subsubsection*{Comparaison D/H}

Pour évaluer la capacité du MCGT à reproduire l’abondance de deutérium, on compare la valeur moyenne modélisée au ratio observé avec sa barre d’erreur :

\begin{figure}[htbp]
  \centering
  \includegraphics[width=0.75\linewidth]{05-nucleosynthese-primordiale/fig_02_dh_modele_vs_obs.png}
  \caption{Comparaison du rapport D/H : Observations vs Modèle MCGT.}
  \label{fig:nucleosynthese_dh_modele_vs_obs}
\end{figure}

{\small Résumé : la valeur modélisée \((D/H)^{\rm mod}\approx2{,}530\times10^{-5}\) se situe bien dans la plage d’incertitude de la mesure \((2{,}527\pm0{,}030)\times10^{-5}\), confirmant la concordance.}

\subsubsection*{Comparaison \(Y_{p}\)}

Le graphique suivant illustre la comparaison de la fraction massique d’hélium \(^4\) :

\begin{figure}[htbp]
  \centering
  \includegraphics[width=0.75\linewidth]{05-nucleosynthese-primordiale/fig_03_yp_modele_vs_obs.png}
  \caption{Comparaison de la fraction massique \(Y_{p}\) : Observations vs Modèle MCGT.}
  \label{fig:nucleosynthese_yp_modele_vs_obs}
\end{figure}

{\small Résumé : la valeur modélisée \(Y_{p}^{\rm mod}\approx0{,}2451\) se trouve au centre de l’intervalle observé \(0{,}2449\pm0{,}0040\), validant également cette abondance.}

\subsubsection*{Évolution de \(\chi^2_{\rm BBN}(T)\)}

Pour visualiser la qualité d’ajustement sur toute la plage de températures, on trace la fonction de coût :

\begin{figure}[htbp]
  \centering
  \includegraphics[width=0.85\linewidth]{05-nucleosynthese-primordiale/fig_04_chi2_nucleosynthese_vs_t.png}
  \caption{Évolution de \(\chi^2_{\rm BBN}(T)\) en fonction de l’âge cosmique \(T\) (échelle logarithmique) ; ligne repère \(\chi^{2}=0{,}01\) et indication du minimum.}
  \label{fig:chi2_nucleosynthese_vs_t}
\end{figure}

{\small Résumé : \(\chi^2_{\rm BBN}(T)\) reste de l’ordre de \(1{,}06\times10^{-2}\) sur toute la période critique, avec un minimum \(\chi^2\approx1{,}06\times10^{-2}\), attestant de la robustesse de la modélisation.}

\subsection{Récapitulatif des scripts}

Le répertoire comporte trois scripts principaux pour automatiser le workflow BBN :

\begin{description}
  \item[\texttt{13\_executer\_alterbbn\_plateau.py}]
    Convertit \texttt{02\_pdot\_tableau\_plateau\_fine.dat} en \texttt{bbn\_input.txt} (format AlterBBN), avec vérification de la validité numérique des ratios.

  \item[\texttt{13\_calculer\_chi2\_nucleosynthese.py}]
    Lit \texttt{05\_nucleosynthese\_resultats\_exact.csv}, calcule \(\chi^{2}_{\rm BBN}(T)\) selon la section « Définition de la fonction de coût \(\chi^{2}_{\rm BBN}(T)\) », et écrit \texttt{05\_chi2\_nucleosynthese\_vs\_t.csv}.

  \item[\texttt{13\_tracer\_resultats\_nucleosynthese.py}]
    Trace et sauve les figures comparatives (D/H, \(Y_{p}\), \(\chi^{2}\)) à partir des CSV générés.
\end{description}

\noindent\textbf{Dépendances} : Python 3.x, \texttt{pandas}, \texttt{matplotlib}.

\subsection*{Glossaire opérationnel}

\begin{description}
  \item[\(\dot P(T)\)] Dérivée propre du temps, extraite de \texttt{02\_pdot\_tableau\_plateau\_fine.dat}.
  \item[\(r(T)\)] Ratio d’expansion, défini comme \(r(T)=1/\dot P(T)\).
  \item[\texttt{bbn\_input.txt}] Fichier d’entrée pour AlterBBN au format :
    \(\{T\,(\mathrm{s}),\,r(T)\}\).
  \item[\texttt{05\_nucleosynthese\_resultats\_exact.csv}] Sortie d’AlterBBN, colonnes :
    \(T\,[\mathrm{Gyr}],\,Y_{p}^{\rm mod}(T),\,(D/H)^{\rm mod}(T)\).
  \item[\texttt{05\_chi2\_nucleosynthese\_vs\_t.csv}] Résultat du calcul de \(\chi^{2}_{\rm BBN}(T)\),
    colonnes : \(T\,[\mathrm{Gyr}],\,\chi^{2}_{\rm BBN}(T)\).
  \item[\(\chi^{2}_{\rm BBN}(T)\)] Fonction de coût mesurant l’écart au carré des abondances modélisées
    par rapport aux valeurs observées.
  \item[\texttt{13\_executer\_alterbbn\_plateau.py}] Script de conversion de la table \(\dot P(T)\) en
    \texttt{bbn\_input.txt}.
  \item[\texttt{13\_calculer\_chi2\_nucleosynthese.py}] Script de lecture de \texttt{05\_nucleosynthese\_resultats\_exact.csv},
    calcul de \(\chi^{2}_{\rm BBN}(T)\) et écriture de \texttt{05\_chi2\_nucleosynthese\_vs\_t.csv}.
  \item[\texttt{13\_tracer\_resultats\_nucleosynthese.py}] Script de génération des figures comparatives D/H, \(Y_{p}\) et
    \(\chi^{2}\) à partir des fichiers CSV.
\end{description}

\bigskip
\noindent\emph{Fin de la partie détaillée, Chapitre 5.}
