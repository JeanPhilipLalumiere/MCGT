\subsection{Définitions mathématiques complètes}

\begin{align}
  \frac{\Delta C_{\ell}}{C_{\ell}}
  &= \frac{C_{\ell}^{\mathrm{MCGT}} - C_{\ell}^{\Lambda\rm CDM}}
           {C_{\ell}^{\Lambda\rm CDM}},
  & 2 \le \ell \le 2500, \\[6pt]
  \frac{\Delta r_{s}}{r_{s}}
  &= \frac{r_{s}^{\mathrm{MCGT}} - r_{s}^{\Lambda\rm CDM}}
           {r_{s}^{\Lambda\rm CDM}}, \\[6pt]
  \Delta\chi^{2}_{\rm Planck}
  &= \chi^{2}_{\rm Planck}(\mathrm{MCGT})
   - \chi^{2}_{\rm Planck}(\Lambda\mathrm{CDM}).
\end{align}

\noindent
Où :
\begin{itemize}
  \item \(C_{\ell}^{X}\) est le spectre CMB calculé pour le modèle \(X\).  
  \item \(r_{s}^{X}\) est l’échelle acoustique à recombinaison pour \(X\).  
  \item \(\chi^{2}_{\rm Planck}(X)\) est la vraisemblance Planck évaluée pour \(X\).
\end{itemize}

\subsection{Spécifications du fichier \texttt{06\_delta\_rs\_scan\_complet.csv}}

Le fichier paramétrique complet contient trois colonnes, dans l’ordre :
\begin{center}
\begin{tabular}{l l l}
\toprule
\textbf{Colonne}     & \textbf{Type} & \textbf{Description}                               \\
\midrule
\texttt{param\_name} & chaîne        & Nom du paramètre varié (\texttt{alpha1}, \texttt{T\_c}, \texttt{Delta}) \\
\texttt{param\_value}& réel          & Valeur numérique du paramètre                     \\
\texttt{delta\_rs\_over\_rs} & réel    & Écart relatif \(\Delta r_{s}/r_{s}\)                \\
\bottomrule
\end{tabular}
\end{center} 

\subsection{Calcul détaillé de $H_{\mathrm{MCGT}}/H_{\Lambda\mathrm{CDM}}$}
\[
  \frac{H_{\mathrm{MCGT}}(z)}{H_{\Lambda\mathrm{CDM}}(z)}
    = \frac{1}{\dot P\bigl(T(z)\bigr)},
  \quad
  T(z) = \int_{z}^{\infty}\frac{dz'}{(1+z')\,H_{\Lambda\mathrm{CDM}}(z')}.
\]

\subsection{Contenu de \texttt{06\_planck\_likelihood.ini}}
Le fichier:
\begin{verbatim}
[CAMBparams]
# Fichier de base des paramètres cosmologiques
paramfile = params.ini

# Fond modifié par MCGT
background_file = 06_hubble_mcgt.dat

# Paramètres de précision
lmax_scalar        = 2500
accuracy_level     = 1
k_eta_max_scalar   = 10000

# Planck likelihoods (TT/TE/EE/lensing)
use_planck_likelihood = T
planck_data_dir       = ../data/planck/plik_lite_TTTEEE/
planck_likelihood     = plik_lite_TTTEEE.clik
\end{verbatim}
\noindent\emph{Remarques à vérifier avant exécution :}
\begin{itemize}
  \item \texttt{params.ini} doit contenir les constantes de base
    \(\Omega_b,\;\Omega_c,\;H_0,\;\tau,\dots\).
  \item \texttt{planck\_data\_dir} doit pointer vers le dossier local
    contenant les fichiers \texttt{.clik} de la likelihood Planck.
\end{itemize}

\subsection{Description pas-à-pas de \texttt{13\_executer\_cmb\_complet.py}}

Le script \texttt{13\_executer\_cmb\_complet.py} orchestre l’ensemble du workflow CMB selon les étapes suivantes :

\begin{enumerate}[label=\arabic*.]
  \item \textbf{Chargement des données}  
    Lit le fichier \texttt{06\_hubble\_mcgt.dat} contenant 
    \(\{z,\;H_{\mathrm{MCGT}}(z)/H_{\Lambda\mathrm{CDM}}(z)\}\).
  \item \textbf{Appel de CAMB}  
    Exécute CAMB deux fois :  
    \begin{itemize}
      \item \verb|camb 06_planck_likelihood.ini --paramset=LCDM|  
      \item \verb|camb 06_planck_likelihood.ini --paramset=MCGT|
    \end{itemize}
  \item \textbf{Calcul des écarts spectres}  
    Calcule \(\Delta C_{\ell}/C_{\ell}\) pour chaque \(\ell\) en comparant  
    \(\{C_{\ell}^{\Lambda\mathrm{CDM}}\}\) et \(\{C_{\ell}^{\mathrm{MCGT}}\}\).
  \item \textbf{Extraction des pics acoustiques}  
    Détermine les indices \(\ell_{1},\ell_{2},\ell_{3}\) par interpolation  
    autour des maxima locaux de \(\ell(\ell+1)C_{\ell}\).
  \item \textbf{Calcul de l’écart acoustique}  
    Mesure \(\Delta r_{s}/r_{s}\) pour chaque configuration paramétrique.
  \item \textbf{Évaluation de \(\Delta\chi^{2}_{\mathrm{Planck}}\)}  
    Compare les vraisemblances Planck pour LCDM et MCGT.
  \item \textbf{Écriture du fichier résultat}  
    Concatène tous les résultats dans  
    \texttt{06\_cmb\_resultats\_complets.csv}.
\end{enumerate}

\subsection{Algorithme d’extraction des pics acoustiques}

Pour chaque spectre \(\{C_{\ell}^{\mathrm{MCGT}}\}\), le script procède ainsi :

\begin{enumerate}[label=\arabic*.]
  \item \textbf{Définition des fenêtres de recherche}  
    Autour des positions théoriques des trois premiers pics :
    \(\ell \approx 220,\;545,\;800\), on fixe des intervalles \(\pm\Delta\ell\) (typiquement \(\pm10\)).
  \item \textbf{Détection brute des maxima}  
    Dans chaque intervalle, repérer l’indice \(\ell_{\max}^{(k)}\) où \(\ell(\ell+1)C_{\ell}\) est maximal.
  \item \textbf{Interpolation quadratique}  
    Considérer les points \(\ell_{\max}^{(k)} - 1,\;\ell_{\max}^{(k)},\;\ell_{\max}^{(k)} + 1\) et ajuster un polynôme du second degré pour affiner la position du pic \(\ell_{k}\) à une précision d’un demi-entier.
  \item \textbf{Enregistrement}  
    Stocker \(\ell_{1},\,\ell_{2},\,\ell_{3}\) dans les colonnes dédiées de \texttt{06\_cmb\_resultats\_complets.csv}.
\end{enumerate}

\subsection{Calcul de \(\Delta r_{s}/r_{s}\)}
L’échelle acoustique à la recombinaison \(r_{s}\) est donnée par :
\[
  r_{s}^{X} 
  = 
  \int_{z_{\mathrm{rec}}}^{\infty} \frac{c_{s}(z)}{H^{X}(z)}\,dz,
  \quad
  z_{\mathrm{rec}}\approx1090,
\]
où \(X=\Lambda\mathrm{CDM}\) ou \(X=\mathrm{MCGT}\).  
Pour chaque triple paramétrique \((\alpha_{1},\,T_{c},\,\Delta)\) :
\begin{enumerate}
  \item Générer \(H_{\mathrm{MCGT}}(z)\) via CAMB (utilisant \texttt{06\_hubble\_mcgt.dat})
        ou, si disponible, via une évaluation analytique de \(\dot P[T(z)]\).
  \item Calculer numériquement
        \(\displaystyle r_{s}^{\mathrm{MCGT}} 
         = \int_{1090}^{\infty} c_{s}(z)\,/\,H_{\mathrm{MCGT}}(z)\,dz.\)
  \item Obtenir \(r_{s}^{\Lambda\mathrm{CDM}}\) à partir d’un appel CAMB en mode LCDM.
  \item Déterminer
        \(\displaystyle
          \Delta r_{s}/r_{s}
          = 
          \bigl(r_{s}^{\mathrm{MCGT}} - r_{s}^{\Lambda\mathrm{CDM}}\bigr)
/\,r_{s}^{\Lambda\mathrm{CDM}}.\)
  \item Enregistrer \((\alpha_{1},\,T_{c},\,\Delta,\;\Delta r_{s}/r_{s})\) 
        dans \texttt{06\_delta\_rs\_scan.csv}.
\end{enumerate}
Le fichier \texttt{06\_delta\_rs\_scan.csv} est ensuite utilisé par 
\texttt{13\_tracer\_delta\_rs.py} pour tracer 
\texttt{fig\_03\_delta\_rs\_vs\_params.png}.

\subsection{Calcul de \(\Delta\chi^{2}_{\mathrm{Planck}}\)}
Après exécution de CAMB en mode LCDM et MCGT avec likelihood Planck activée, on récupère :
\[
  \chi^{2}_{\mathrm{Planck}}(\Lambda\mathrm{CDM}), 
  \quad
  \chi^{2}_{\mathrm{Planck}}(\mathrm{MCGT}).
\]
Le script \texttt{13\_executer\_cmb\_complet.py} calcule alors :
\[
  \Delta\chi^{2}_{\mathrm{Planck}}
  \;=\;
  \chi^{2}_{\mathrm{Planck}}(\mathrm{MCGT})
  \;-\;
  \chi^{2}_{\mathrm{Planck}}(\Lambda\mathrm{CDM}),
\]
et l’ajoute à la colonne correspondante de 
\texttt{06\_cmb\_resultats\_complets.csv}.

\subsection{Structure de \texttt{06\_cmb\_resultats\_complets.csv}}

Chaque ligne du fichier correspond à une configuration paramétrique \((\alpha_{1},\,T_{c},\,\Delta)\) et contient neuf colonnes, dans l’ordre :

\begin{center}
\begin{tabular}{l l l}
\toprule
\textbf{Colonne}               & \textbf{Type} & \textbf{Description}                                                           \\
\midrule
\texttt{alpha1}                & réel          & Paramètre \(\alpha_{1}\)                                                       \\
\texttt{T\_c}                  & réel          & Température critique \(T_{c}\) [Gyr]                                          \\
\texttt{Delta}                 & réel          & Durée \(\Delta\) [Gyr]                                                         \\
\texttt{delta\_rs\_over\_rs}   & réel          & Écart acoustique \(\Delta r_{s}/r_{s}\)                                        \\
\texttt{max\_delta\_C\_over\_C} & réel         & \(\max_{\ell}|\Delta C_{\ell}/C_{\ell}|\)                                       \\
\texttt{ell\_1}                & entier        & Premier pic acoustique \(\ell_{1}\)                                            \\
\texttt{ell\_2}                & entier        & Deuxième pic acoustique \(\ell_{2}\)                                           \\
\texttt{ell\_3}                & entier        & Troisième pic acoustique \(\ell_{3}\)                                          \\
\texttt{delta\_chi2\_planck}   & réel          & \(\Delta\chi^{2}_{\mathrm{Planck}}\)                                           \\
\bottomrule
\end{tabular}
\end{center}

\noindent\emph{Chaque valeur numérique est enregistrée en virgule flottante (6 chiffres significatifs), sauf les indices \(\ell_{k}\) en entier.}

\subsection{Sources de données complémentaires}

Pour enrichir l’analyse, deux fichiers de scan paramétrique sont désormais disponibles :

\begin{itemize}
  \item \texttt{06-rayonnement-cmb/06\_cmb\_resultats\_scan\_chi2.csv}  
        contient les colonnes \texttt{alpha1}, \texttt{T\_c}, \texttt{Delta} et \texttt{Delta\_chi2\_Planck}.  
        Chaque ligne donne la valeur de \(\Delta\chi^2_{\rm Planck}\) pour une configuration \((\alpha_{1},T_{c},\Delta)\),  
        et est générée par \texttt{13\_generer\_cmb\_scan\_chi2.py}.
  \item \texttt{06-rayonnement-cmb/06\_delta\_rs\_scan\_complet.csv}  
        contient les colonnes \texttt{param\_name}, \texttt{param\_value} et \texttt{delta\_rs\_over\_rs}.  
        Permet d’étudier la dépendance de \(\Delta r_{s}/r_{s}\) à la variation unidimensionnelle de chacun des paramètres \(\alpha_{1}\), \(T_{c}\) ou \(\Delta\),  
        et est généré par \texttt{13\_generer\_delta\_rs\_scan\_complet.py}.
\end{itemize}

\subsection*{Lien vers les données Planck}

Les données Planck utilisées (version 2018) sont accessibles à l’adresse suivante :  
\url{https://pla.esac.esa.int/pla/#home}

\subsection{Analyse de sensibilité}

\begin{figure}[htbp]
  \centering
  \includegraphics[width=0.75\linewidth]{06-rayonnement-cmb/fig_05_carte_chaleur_delta_chi2.png}
  \caption{Heatmap de \(\Delta\chi^2_{\rm Planck}\) selon \(\alpha_{1}\) et \(T_c\). Les zones sombres indiquent une meilleure concordance avec les données Planck.}
  \label{fig:carte_chaleur_delta_chi2}
\end{figure}

\noindent
La Fig.~\ref{fig:carte_chaleur_delta_chi2} met en évidence la région optimale du plan \((\alpha_{1},T_c)\) où \(\Delta\chi^2_{\rm Planck}\) est minimal (autour de \(\alpha_{1}\approx0{,}3\), \(T_c\approx0{,}3\) Gyr).  

\medskip
\noindent
Une analyse unidimensionnelle à partir de \texttt{06\_delta\_rs\_scan\_complet.csv} montre par exemple que,  
pour une augmentation de \(\alpha_{1}\) de +0,05 (passant de 0,30 à 0,35),  
l’écart acoustique \(\Delta r_{s}/r_{s}\) évolue d’environ \(-1\times10^{-5}\) à \(0\),  
soit une variation d’environ \(1\times10^{-5}\).  

\subsection*{Encadré : Limitations \& Perspectives}

\begin{itemize}
  \item Dépendance à l’interpolation de la fonction \(H_{\Lambda\mathrm{CDM}}(z)\),  
        qui peut introduire des erreurs de discrétisation aux hauts redshifts.  
  \item Absence d’effets non-linéaires dans le calcul des spectres (petits droits de lensing non inclus),  
        susceptibles d’affiner légèrement les positions et amplitudes des pics.  
  \item Perspectives d’extension :
    \begin{itemize}
      \item Inclusion de la polarisation CMB (TE/EE) pour renforcer la contrainte sur \(\Delta\chi^2\).  
      \item Modélisation des effets de lentilles gravitationnelles sur le spectre à haut \(\ell\).  
      \item Exploration de couplages supplémentaires (e.g. neutrinos massifs, modifications de la recombinaison) pour tester la robustesse du MCGT.
    \end{itemize}
\end{itemize}

\subsection{Reproductibilité}

Pour reproduire tous les résultats du test CMB, exécuter successivement :
\begin{enumerate}
  \item \texttt{13\_executer\_cmb\_complet.py}
  \item \texttt{13\_valider\_resultats\_cmb.py}
  \item \texttt{13\_tracer\_delta\_rs.py}
\end{enumerate}

\subsection*{Glossaire}

\begin{description}
  \item[\(\alpha_{1}\)] Coefficient du couplage logistique modulant la transition du modèle MCGT.
  \item[\(T_{c}\)] Température critique (en Gyr) caractérisant le changement de régime.
  \item[\(\Delta\)] Durée de la transition (en Gyr) autour de \(T_{c}\).
  \item[\(\dot P(T)\)] Dérivée calibrée du paramètre \(P\) en fonction de la température \(T\), utilisée pour calculer \(\frac{H_{\mathrm{MCGT}}}{H_{\Lambda\mathrm{CDM}}}\).
  \item[\(\tfrac{H_{\mathrm{MCGT}}(z)}{H_{\Lambda\mathrm{CDM}}(z)}\)] Ratio des taux d’expansion MCGT et \(\Lambda\)CDM en fonction du redshift \(z\).
  \item[\(C_{\ell}\)] Spectre de puissance angulaire du fond diffus cosmologique pour le multipôle \(\ell\).
  \item[\(\Delta C_{\ell}/C_{\ell}\)] Écart relatif entre les spectres MCGT et \(\Lambda\)CDM.
  \item[\(r_{s}\)] Échelle acoustique à la recombinaison.
  \item[\(\Delta r_{s}/r_{s}\)] Écart relatif de l’échelle acoustique entre MCGT et \(\Lambda\)CDM.
  \item[\(\chi^{2}_{\mathrm{Planck}}\)] Fonction de vraisemblance Planck utilisée pour l’évaluation du modèle.
  \item[\(\Delta\chi^{2}_{\mathrm{Planck}}\)] Différence de \(\chi^{2}_{\mathrm{Planck}}\) entre MCGT et \(\Lambda\)CDM.
  \item[\texttt{13\_executer\_cmb\_complet.py}] Script principal orchestrant l’ensemble du pipeline CMB.
  \item[\texttt{13\_valider\_resultats\_cmb.py}] Script calculant et visualisant \(\Delta C_{\ell}/C_{\ell}\).
  \item[\texttt{13\_tracer\_delta\_rs.py}] Script traçant \(\Delta r_{s}/r_{s}\) en fonction des paramètres.
  \item[\texttt{06\_cmb\_resultats\_complets.csv}] Fichier résultant regroupant tous les écarts numériques et les indices de pics acoustiques.
  \item[\texttt{06\_delta\_rs\_scan\_complet.csv}] Fichier listant les écarts acoustiques pour chaque variation unidimensionnelle de paramètre.
  \item[\texttt{06\_planck\_likelihood.ini}] Fichier de configuration CAMB pour l’évaluation de la likelihood Planck.
\end{description}

\bigskip
\noindent\emph{Fin de la partie détaillée, Chapitre 6.}