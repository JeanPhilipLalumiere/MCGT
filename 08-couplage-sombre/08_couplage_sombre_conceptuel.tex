\section{Chapitre 8 – Couplage sombre}

\subsection{Modèle d’interaction}

Nous considérons, au niveau conceptuel, un couplage modéré entre matière sombre et énergie sombre, caractérisé par un unique paramètre \(Q_{0}\).

Les densités aux temps présents sont fixées par :
\[
  \rho_{m}(0) = \rho_{c,0}\,\Omega_{m,0},
  \quad
  \rho_{\phi}(0) = \rho_{c,0}\,\Omega_{\phi,0},
  \quad
  (\Omega_{m,0},\,\Omega_{\phi,0}) = (0.315,\,0.685).
\]

Le paramètre d’interaction \(Q_{0}\) contrôle le transfert d’énergie entre les deux composantes et est exploré sur l’intervalle
\[
  Q_{0} \in [0,\,0.20].
\]

\subsection{Jeux de données externes}

Pour contraindre le paramètre \(Q_{0}\), on s’appuie sur trois ensembles de données :

\begin{itemize}
  \item \textbf{Supernovae Ia – Pantheon\,+ v2.0 (Scolnic et al. 2022)}
    \begin{itemize}
      \item Fichier : \texttt{08\_pantheon\_plus.csv}
      \item Observables : module de distance \(\mu_{\rm obs}(z)\) et incertitude \(\sigma_{\mu}(z)\) pour 1550 SNIa.
    \end{itemize}

  \item \textbf{Oscillations acoustiques baryoniques – BAO DR12/eBOSS (Alam et al. 2021)}
    \begin{itemize}
      \item Fichier : \texttt{08\_donnees\_bao.csv}
      \item Observables : distance volumétrique \(D_{V}^{\rm obs}(z)\) et incertitude \(\sigma_{D_{V}}(z)\) pour 8 mesures BAO.
    \end{itemize}

  \item \textbf{Contrainte locale sur la constante de Hubble}
    \[
      H_{0}^{\rm local} = 73.2 \pm 1.3 \;\mathrm{km/s/Mpc}
      \quad(\text{Riess et al. 2021}).
    \]
    La covariance utilisée dans les scripts est \(\sigma_{H_0}=1.3\;\mathrm{km/s/Mpc}\).
\end{itemize}

\subsection{Outils d’implémentation}

La chaîne de calcul repose sur trois scripts clés, dont l’exécution produit les fichiers de résultats listés dans ce chapitre :

\begin{itemize}
  \item \texttt{13\_resolution\_grille\_z.py}
    \begin{itemize}
      \item Résout numériquement l’évolution de \(H(z;Q_{0})\) pour chaque valeur de \(Q_{0}\).
      \item Génère la grille ASCII \texttt{08\_grille\_Hz\_q0.txt}, listant \(\{z_{i},\,H(z_{i};Q_{0})\}\).
    \end{itemize}

  \item \texttt{13\_couplage\_energie\_sombre.py}
    \begin{itemize}
      \item Parcourt l’intervalle \(Q_{0}\in[0,\,0.20]\).
      \item Pour chaque \(Q_{0}\) :
        \begin{itemize}
          \item appelle \texttt{13\_resolution\_grille\_z.py},
          \item calcule conceptuellement les distances \(D_{L}\) et \(D_{V}\) ainsi que les contributions \(\chi^{2}_{H_{0}}\), \(\chi^{2}_{\rm SNIa}\), \(\chi^{2}_{\rm BAO}\),
          \item constitue la statistique globale \(\chi^{2}_{\rm tot}\),
          \item écrit la ligne correspondante dans \texttt{08\_q0\_reel\_scan.csv}.
        \end{itemize}
      \item Procède ensuite à une interpolation quadratique locale pour estimer
        \(Q_{0}^{\star}\pm\sigma\) et sauvegarde ce résultat dans
        \texttt{08\_q0\_meilleur\_ajustement.txt}.
    \end{itemize}

  \item \texttt{13\_valider\_resultats\_couplage.py}
    \begin{itemize}
      \item Lit \texttt{08\_q0\_reel\_scan.csv} et génère trois figures conceptuelles :
        \begin{itemize}
          \item \texttt{fig\_01\_chi2\_total\_vs\_q0.png} : \(\chi^{2}_{\rm tot}(Q_{0})\),
          \item \texttt{fig\_02\_dv\_vs\_z.png} : comparaison BAO \(D_{V}^{\rm obs}\) vs \(D_{V}^{\rm th}\),
          \item \texttt{fig\_03\_mu\_vs\_z.png} : comparaison Pantheon+ \(\mu_{\rm obs}\) vs \(\mu_{\rm th}\).
        \end{itemize}
    \end{itemize}
\end{itemize}

\subsection{Fichiers de sortie clés}

La chaîne de calcul génère les fichiers et figures suivants, essentiels pour l’interprétation des résultats :

\begin{itemize}
  \item \texttt{08\_grille\_Hz\_q0.txt}
    Grille de redshift et de taux de Hubble \(\{z_{i},H(z_{i};Q_{0})\}\) issue du script \texttt{13\_resolution\_grille\_z.py}.
  \item \texttt{08\_q0\_reel\_scan.csv}
    Tableau complet des statistiques partielles et globale
    \((Q_{0},\chi^{2}_{H_{0}},\chi^{2}_{\rm SNIa},\chi^{2}_{\rm BAO},\chi^{2}_{\rm tot})\).
  \item \texttt{08\_q0\_meilleur\_ajustement.txt}
    Valeur optimale \(Q_{0}^{\star}\pm\sigma\) obtenue par interpolation locale autour du minimum de \(\chi^{2}_{\rm tot}\).
  \item \texttt{fig\_01\_chi2\_total\_vs\_q0.png}
    Trace de la courbe \(\chi^{2}_{\rm tot}(Q_{0})\), utilisée pour identifier le couplage optimal.
  \item \texttt{fig\_02\_dv\_vs\_z.png}
    Superposition du modèle et des données BAO pour la distance volumétrique \(D_{V}(z)\), vérifiant la concordance.
  \item \texttt{fig\_03\_mu\_vs\_z.png}
    Superposition du modèle et des données Pantheon\,+ pour le module de distance \(\mu(z)\), contrôlant l’accord avec les SN Ia.
\end{itemize}

\subsection{Intervalle exploré pour \(Q_{0}\)}

Le paramètre d’interaction \(Q_{0}\) fait l’objet d’un balayage sur l’intervalle suivant :
\[
  Q_{0}\in[0,\,0.20].
\]

\subsection{Workflow conceptuel}

La procédure conceptuelle se résume en quatre étapes clés :

\begin{enumerate}
  \item \emph{Balayage de l’intervalle \(Q_{0}\).} Pour chaque valeur de \(Q_{0}\in[0,\,0.20]\), on résout numériquement l’évolution de \(H(z;Q_{0})\) sur la plage de redshift considérée.
  \item \emph{Calcul des distances cosmologiques.} À partir du profil \(H(z;Q_{0})\), on déduit les distances lumineuse et volumétrique \(D_{L}(z;Q_{0})\) et \(D_{V}(z;Q_{0})\).
  \item \emph{Évaluation des statistiques de coût.} On compare ces quantités aux données pour calculer
    \(\chi^{2}_{H_{0}},\,\chi^{2}_{\rm SNIa},\,\chi^{2}_{\rm BAO}\),
    puis on constitue la statistique globale
    \[
      \chi^{2}_{\rm tot}(Q_{0})
      = \chi^{2}_{H_{0}}(Q_{0})
      + \chi^{2}_{\rm SNIa}(Q_{0})
      + \chi^{2}_{\rm BAO}(Q_{0}).
    \]
  \item \emph{Extraction du meilleur couplage.} Enfin, on identifie \(Q_{0}^{\star}\) par interpolation locale autour du minimum de \(\chi^{2}_{\rm tot}\) et l’on trace la courbe \(\chi^{2}_{\rm tot}(Q_{0})\).
\end{enumerate}

\subsection{Résultats graphiques}

\begin{figure}[htbp]
  \centering
  \includegraphics[width=0.75\linewidth]%
    {08-couplage-sombre/fig_01_chi2_total_vs_q0.png}
  \caption{Variation de la statistique normalisée \(\chi^{2}_{\rm tot}(Q_{0})\) en fonction du paramètre de couplage.
  Chaque point correspond à une valeur de \(Q_{0}\) dans \([0,0.20]\). Le minimum localisé à \(Q_{0}^{\star}=0{,}12\) donne \(\chi^{2}_{\rm tot}/N\approx1.00\).}
  \label{fig:chi2_total_vs_Q0}
\end{figure}

\begin{figure}[htbp]
  \centering
  \includegraphics[width=0.75\linewidth]%
    {08-couplage-sombre/fig_02_dv_vs_z.png}
  \caption{Comparaison de la distance volumétrique
    \(D_{V}(z)\) prédite par le modèle (courbe orange, \(Q_{0}^{\star}=0{,}12\))
    aux mesures BAO (points verts avec barres d’erreur).
    On observe une bonne concordance pour les quatre redshifts clés.}
  \label{fig:DV_z_p4}
\end{figure}

\begin{figure}[htbp]
  \centering
  \includegraphics[width=0.75\linewidth]%
    {08-couplage-sombre/fig_03_mu_vs_z.png}
  \caption{Module de distance \(\mu(z)\) : modèle (courbe rouge, \(Q_{0}^{\star}=0{,}12\))
    versus données Pantheon\,+ (points bleus).
    Le couplage sombre n’introduit pas de décalage systématique hors des barres d’erreur SN Ia.}
  \label{fig:mu_z_p4}
\end{figure}

\subsection{Conclusion conceptuelle}

Le couplage optimal pour l’interaction matière sombre / énergie sombre est estimé à
\[
  Q_{0}^{\star} \approx 0{,}12,
\]
avec une statistique globale normalisée \(\chi^{2}_{\rm tot}/N \approx 1\).

Cette valeur de \(Q_{0}\) est pleinement compatible, au niveau conceptuel, avec les observations Supernovae Ia, BAO et la contrainte locale sur \(H_{0}\).

\subsection{Impact sur la tension de Hubble}

La modification MCGT de l’expansion précoce, via la calibration logistique et le couplage sombre, induit une petite correction à la constante de Hubble au moment de la recombinaison :
\[
  \frac{\delta H_0}{H_0}
  \sim \mathcal{O}(1\%)\,.
\]

\begin{itemize}
  \item Valeurs de référence en \(\Lambda\)CDM :
    \(H_0^{\rm Planck} = 67.4 \pm 0.5\;\mathrm{km/s/Mpc}\)\;(\text{Planck 2018}),
    \(H_0^{\rm local} = 73.2 \pm 1.3\;\mathrm{km/s/Mpc}\)\;(\text{Riess et al. 2021}),
    soit une tension de \(\sim6\%\).
  \item Estimation MCGT :
    en appliquant la correction logistique (voir Sect. 1.2) et le couplage sombre (Chap. 8), on calcule
    \(\delta H_0\) via l’évolution de \(H(z)\) autour du dernier scattering — détails dans la partie “Formules des distances cosmologiques” et “Formules des statistiques \(\chi^2\)”.
  \item Résultat :
    MCGT réduit la tension de l’ordre de quelques pour‐cents, ramenant partiellement \(H_0^{\rm rec}\) vers la valeur locale,
    mais sans l’éliminer complètement.
  \item Perspectives :
    l’ajout de couplages additionnels ou de nouveaux degrés de liberté (ex. couplage avec neutrinos, champs scalaires dynamiques)
    pourrait amplifier \(\delta H_0\) et tendre vers une concordance complète.
\end{itemize}

\noindent\emph{Fin du volet conceptuel du Chapitre 8. La partie opérationnelle détaillée commence ci-dessous.}
