\subsection{Équations complètes du couplage}

Le système détaillé décrivant l’échange d’énergie entre matière sombre et énergie sombre s’écrit :

\[
  \begin{cases}
    \dot{\rho}_{m} + 3H\,\rho_{m}
      = -\,Q_{0}\,H_{0}\,\rho_{m},\\[6pt]
    \dot{\rho}_{\phi} + 3H\,(1 + w_{\phi})\,\rho_{\phi}
      = +\,Q_{0}\,H_{0}\,\rho_{m},
    \quad w_{\phi} = -1.
  \end{cases}
\]

\subsection{Conditions initiales détaillées}

Les densités aux temps présents sont fixées par :
\[
  \rho_{m}(0) = \rho_{c,0}\,\Omega_{m,0}, 
  \quad
  \rho_{\phi}(0) = \rho_{c,0}\,\Omega_{\phi,0},
  \quad
  (\Omega_{m,0},\,\Omega_{\phi,0}) = \bigl(0{,}315,\,0{,}685\bigr).
\]

\subsection{Formules des distances cosmologiques}

Les distances lumineuse et volumétrique se formalisent par :
\[
  D_{L}(z) \;=\; (1+z)\int_{0}^{z} \frac{c\,dz'}{H(z')},
  \qquad
  D_{V}(z) \;=\; \Biggl[\frac{c\,z}{H(z)}\Bigl(\int_{0}^{z}\frac{c\,dz'}{H(z')}\Bigr)^{2}\Biggr]^{1/3}.
\]

\subsection{Formules des statistiques \(\chi^{2}\)}

Les statistiques de coût se formalisent par :
\[
  \chi^{2}_{\rm SNIa}
  = \sum_{i}
    \frac{\bigl[\mu_{\rm th}(z_{i}) - \mu_{\rm obs}(z_{i})\bigr]^{2}}
         {\sigma_{\mu,i}^{2}},
  \qquad
  \chi^{2}_{\rm BAO}
  = \sum_{j}
    \frac{\bigl[D_{V}(z_{j}) - D_{V}^{\rm obs}(z_{j})\bigr]^{2}}
         {\sigma_{D_{V},j}^{2}},
  \qquad
  \chi^{2}_{H_{0}}
  = \frac{\bigl[H(0) - H_{0}^{\rm loc}\bigr]^{2}}
         {\sigma_{H_{0}}^{2}}.
\]

\subsection{Gestion des alertes}

Tout écart supérieur à \(3\sigma\) doit déclencher une notification au format suivant :

\begin{verbatim}
[ALERTE] Donnée <type> à z=<valeur> hors 3σ : obs=<valeur_obs>, th=<valeur_th>
\end{verbatim}

Par exemple :

\begin{verbatim}
[ALERTE] Donnée BAO à z=0.35 hors 3σ : D_V_obs=1400.2, D_V_th=1380.5
\end{verbatim}

\subsection{Vérifications et cohérence interne}

\begin{itemize}
  \item \textbf{Contrôle du rapport global}  
    Calculer  
    \[
      \frac{\chi^{2}_{\rm tot}(Q_{0}^{\star})}{N_{\rm data}}
      \approx 1,
    \]
    où \(N_{\rm data} = N_{\rm SNIa} + N_{\rm BAO} + 1\) est le nombre total de points analysés.

  \item \textbf{Validation de la contrainte locale \(H_{0}\)}  
    S’assurer que la covariance \(\sigma_{H_{0}} = 1.3\;\mathrm{km/s/Mpc}\) a été correctement appliquée :
    \[
      \chi^{2}_{H_{0}}
      = \frac{\bigl[H(0;Q_{0}^{\star}) - H_{0}^{\rm local}\bigr]^{2}}
             {\sigma_{H_{0}}^{2}}.
    \]

  \item \textbf{Tests ponctuels}  
    \begin{itemize}
      \item Sur un sous-ensemble de Supernovae Ia :  
        \[
          \bigl|\mu_{\rm th}(z_i;Q_{0}^{\star}) - \mu_{\rm obs}(z_i)\bigr|
          < \sigma_{\mu,i}.
        \]
      \item Sur un sous-ensemble de mesures BAO :  
        \[
          \bigl|D_{V}(z_j;Q_{0}^{\star}) - D_{V}^{\rm obs}(z_j)\bigr|
          < \sigma_{D_{V},j}.
        \]
      \item Récupération de \(\Lambda\)CDM (\(Q_{0}=0\)) :  
        \[
          \bigl|H(z;0) - H_{\Lambda\rm CDM}(z)\bigr|
          < 10^{-6}\,H(z).
        \]
    \end{itemize}

  \item \textbf{Gestion des alertes}  
    Tout écart supérieur à \(3\sigma\) doit déclencher une notification au format :
    \begin{verbatim}
[ALERTE] Donnée [type] à z=<valeur> hors 3σ : obs=<valeur_obs>, théorie=<valeur_th>
    \end{verbatim}
\end{itemize}

\subsection{Scripts opérationnels et déroulement}

Les opérations s’effectuent via trois scripts, chaque script gérant à la fois son rôle, les options CLI, les sorties, et les étapes clés du workflow :

\begin{description}
  \item[\texttt{13\_resolution\_grille\_z.py}]  
    \textbf{Rôle :} résoudre numériquement l’évolution de \(H(z;Q_{0})\) conformément aux équations détaillées dans « Équations complètes du couplage ».  
    \textbf{Options CLI :}  
      \verb|--Q0 <float>|, \verb|--Nz <int>|, \verb|--rtol <float>|, \verb|--atol <float>|.  
    \textbf{Déroulement :}
    \begin{enumerate}
      \item Lecture de la valeur du paramètre \(Q_{0}\).  
      \item Intégration numérique de \(H(z)\) sur une grille de \(N_z\) points en \(z\).  
      \item Écriture du fichier de sortie \texttt{08\_grille\_Hz\_q0.txt}, listant les paires \(\{z_{i},\,H(z_{i};Q_{0})\}\).  
    \end{enumerate}

  \item[\texttt{13\_couplage\_energie\_sombre.py}]  
    \textbf{Rôle :} orchestrer le balayage de \(Q_{0}\in[0,0.20]\), calculer distances et \(\chi^2\), puis constituer le scan.  
    \textbf{Options CLI :}  
      \verb|--pantheon <fichier_csv>|,  
      \verb|--bao <fichier_csv>|,  
      \verb|--H0 <float>|,  
      \verb|--sigma_H0 <float>|,  
      \verb|--output_scan <fichier_csv>|,  
      \verb|--output_bestfit <fichier_txt>|.  
    \textbf{Déroulement :}
    \begin{enumerate}
      \item Lecture des fichiers \texttt{08\_pantheon\_plus.csv} et \texttt{08\_donnees\_bao.csv}.  
      \item Boucle pour chaque \(Q_{0}\) de la grille :
        \begin{itemize}
          \item Appel à \texttt{13\_resolution\_grille\_z.py} (génère \texttt{08\_grille\_Hz\_q0.txt}).  
          \item Interpolation spline de \(H(z)\).  
           \item Calcul des distances et des \(\chi^2\) selon les formules données dans les sections « Formules des distances cosmologiques » et « Formules des statistiques \(\chi^2\) ».
          \item Écriture d’une ligne \(\{Q_{0},\chi^{2}_{H_{0}},\chi^{2}_{\rm SNIa},\chi^{2}_{\rm BAO},\chi^{2}_{\rm tot}\}\)  
                dans \texttt{08\_q0\_reel\_scan.csv}.  
        \end{itemize}
      \item Interpolation quadratique locale autour du minimum de \(\chi^{2}_{\rm tot}\)  
            pour estimer \(Q_{0}^{\star}\pm\sigma\), sauvegardé dans \texttt{08\_q0\_meilleur\_ajustement.txt}.  
    \end{enumerate}

  \item[\texttt{13\_valider\_resultats\_couplage.py}]  
    \textbf{Rôle :} générer les diagnostics visuels à partir du scan complet.  
    \textbf{Options CLI :}  
      \verb|--input_scan <fichier_csv>|,  
      \verb|--output_chi2 <fichier_png>|,  
      \verb|--output_dv <fichier_png>|,  
      \verb|--output_mu <fichier_png>|.  
    \textbf{Déroulement :}
    \begin{enumerate}
      \item Lecture de \texttt{08\_q0\_reel\_scan.csv}.  
      \item Tracé de la courbe \(\chi^{2}_{\rm tot}(Q_{0})\) dans \texttt{fig\_01\_chi2\_total\_vs\_q0.png}.  
      \item Superposition modèle vs observations BAO dans \texttt{fig\_02\_dv\_vs\_z.png}.  
      \item Superposition modèle vs SNIa dans \texttt{fig\_03\_mu\_vs\_z.png}.  
    \end{enumerate}
\end{description}

\subsection{Structure des fichiers de sortie}

Les opérations génèrent deux types de fichiers de résultats :

\begin{itemize}
  \item \textbf{Fichier de grille de H(z)}  
    \texttt{08\_grille\_Hz\_q0.txt} : chaque ligne contient deux colonnes séparées par un espace :
    \[
      z_i 
      \quad H(z_i;Q_{0})\,.
    \]

  \item \textbf{Fichier de scan statistique}  
    \texttt{08\_q0\_reel\_scan.csv} : fichier CSV à cinq colonnes, séparées par des virgules, avec en-tête :
    \begin{verbatim}
Q0,chi2_H0,chi2_SNIa,chi2_BAO,chi2_tot
    \end{verbatim}
    Chaque ligne suivante contient pour une valeur de \(Q_{0}\) :
    \[
      Q_{0},\;\chi^{2}_{H_{0}},\;\chi^{2}_{\rm SNIa},\;\chi^{2}_{\rm BAO},\;\chi^{2}_{\rm tot}.
    \]
\end{itemize}

\subsection*{Données théoriques de la courbe optimale}

En complément du fichier de scan statistique, nous stockons :
\begin{itemize}
  \item \texttt{08\_dv\_theorie\_q0star.csv} : valeurs de \(D_V(z_i;Q_0^\star)\) pour chaque redshift \(z_i\).
  \item \texttt{08\_mu\_theorie\_q0star.csv} : valeurs de \(\mu_{\rm th}(z_i;Q_0^\star)\) pour chaque redshift \(z_i\).
\end{itemize}

Ces deux fichiers servent à tracer les figures  
\texttt{fig\_02\_dv\_vs\_z.png} et \texttt{fig\_03\_mu\_vs\_z.png}.

\subsection{Notes d’implémentation}

Le résumé conceptuel (08\_couplage\_sombre\_conceptuel.tex) présente formules, définitions, sources de données, figures et résultats chiffrés.  
Ce document détaille en revanche :

\begin{itemize}
  \item l’arborescence et le rôle des scripts (\texttt{13\_resolution\_grille\_z.py}, \texttt{13\_couplage\_energie\_sombre.py}, \texttt{13\_valider\_resultats\_couplage.py}),  
  \item les formats de fichiers produits (\texttt{08\_grille\_Hz\_q0.txt}, \texttt{08\_q0\_reel\_scan.csv}, \texttt{08\_q0\_meilleur\_ajustement.txt}),  
  \item l’organisation technique des boucles, interpolations et alertes.
\end{itemize}

Pour ajuster l’intervalle de balayage \(Q_{0}\), le maillage en \(z\) ou les tolérances ODE, modifier les constantes en tête de \texttt{13\_resolution\_grille\_z.py} et \texttt{13\_couplage\_energie\_sombre.py}.  

\subsection{Informations complémentaires utiles}

\begin{itemize}
  \item \textbf{Environnement et dépendances}  
    Les scripts ont été testés sous Python 3.10.  
    Les principales bibliothèques requises sont listées dans \texttt{13\_requirements.txt}, notamment :
    \begin{itemize}
      \item \texttt{numpy}  
      \item \texttt{scipy}  
      \item \texttt{pandas}  
      \item \texttt{matplotlib}  
    \end{itemize}

  \item \textbf{Performance / durée}  
    Le balayage complet de \(Q_{0}\) (201 valeurs, résolution ODE sur 1000 points)  
    nécessite environ 3 minutes sur un processeur quad-core standard.

  \item \textbf{Reproductibilité}  
    Pour exécuter l’intégralité de la chaîne de calcul en une seule commande, on peut utiliser un script shell dédié, par exemple :
    \begin{verbatim}
bash run_all_coupling.sh
    \end{verbatim}
    qui appelle successivement :
    \begin{enumerate}
      \item \texttt{13\_resolution\_grille\_z.py}  
      \item \texttt{13\_couplage\_energie\_sombre.py}  
      \item \texttt{13\_valider\_resultats\_couplage.py}
    \end{enumerate}

  \item \textbf{Répertoire de travail}  
    Les scripts doivent être lancés depuis la racine du projet pour que les chemins relatifs  
    (\texttt{08-couplage-sombre/...} et \texttt{13-scripts-annexes/...}) soient correctement résolus.
\end{itemize}

\subsection*{Glossaire}

\begin{description}
  \item[$Q_{0}$] Paramètre d’interaction matière sombre–énergie sombre, détermine le transfert d’énergie entre \(\rho_{m}\) et \(\rho_{\phi}\).
  \item[$\rho_{m}(0),\;\rho_{\phi}(0)$] Densités initiales aujourd’hui, fixées par \(\rho_{c,0}\,\Omega_{m,0}\) et \(\rho_{c,0}\,\Omega_{\phi,0}\).
  \item[$H_{0}^{\rm local}$] Contrainte locale de la constante de Hubble, \(73.2\pm1.3\)\,km/s/Mpc.
  \item[$D_{L}(z)$] Distance lumineuse : 
    \(\displaystyle D_{L}(z) = (1+z)\int_{0}^{z}\frac{c\,dz'}{H(z')}\).
  \item[$D_{V}(z)$] Distance volumétrique :
    \(\displaystyle D_{V}(z) = \Bigl[\frac{c\,z}{H(z)}\Bigl(\int_{0}^{z}\frac{c\,dz'}{H(z')}\Bigr)^{2}\Bigr]^{1/3}.\)
  \item[$\chi^{2}_{H_{0}}$] Contribution du terme \(H_{0}\) : 
    \(\displaystyle \frac{[H(0)-H_{0}^{\rm local}]^{2}}{\sigma_{H_{0}}^{2}}.\)
  \item[$\chi^{2}_{\rm SNIa}$] Statistique Supernovae Ia : 
    \(\displaystyle \sum_{i}\frac{[\mu_{\rm th}(z_{i})-\mu_{\rm obs}(z_{i})]^{2}}{\sigma_{\mu,i}^{2}}.\)
  \item[$\chi^{2}_{\rm BAO}$] Statistique BAO :
    \(\displaystyle \sum_{j}\frac{[D_{V}(z_{j})-D_{V}^{\rm obs}(z_{j})]^{2}}{\sigma_{D_{V},j}^{2}}.\)
  \item[$\chi^{2}_{\rm tot}$] Statistique globale :
    \(\chi^{2}_{\rm tot} = \chi^{2}_{H_{0}} + \chi^{2}_{\rm SNIa} + \chi^{2}_{\rm BAO}.\)
  \item[\texttt{08\_grille\_Hz\_q0.txt}] Grille ASCII : couples \(\{z_{i},H(z_{i};Q_{0})\}\) pour chaque valeur de \(Q_{0}\).
  \item[\texttt{08\_q0\_reel\_scan.csv}] Fichier CSV : colonnes \(Q_{0},\chi^{2}_{H_{0}},\chi^{2}_{\rm SNIa},\chi^{2}_{\rm BAO},\chi^{2}_{\rm tot}\).
  \item[\texttt{08\_q0\_meilleur\_ajustement.txt}] Estimation finale \(Q_{0}^{\star}\pm\sigma\) par interpolation du minimum de \(\chi^{2}_{\rm tot}\).
  \item[\texttt{08\_dv\_theorie\_q0star.csv}] Distances volumétriques théoriques \(D_{V}(z_{i};Q_{0}^{\star})\) pour chaque \(z_{i}\).
  \item[\texttt{08\_mu\_theorie\_q0star.csv}] Modules de distance théoriques \(\mu_{\rm th}(z_{i};Q_{0}^{\star})\) pour chaque \(z_{i}\).
  \item[\texttt{13\_resolution\_grille\_z.py}] Script ODE : solveur de \(H(z;Q_{0})\) et écriture de la grille de redshift.
  \item[\texttt{13\_couplage\_energie\_sombre.py}] Script de balayage : calcule distances, \(\chi^{2}\) et produit le scan (\texttt{08\_q0\_reel\_scan.csv}) et le best‐fit.
  \item[\texttt{13\_valider\_resultats\_couplage.py}] Script de diagnostic : génère les tracés \(\chi^{2}_{\rm tot}(Q_{0})\), \(D_{V}\) vs \(z\) et \(\mu\) vs \(z\).
\end{description}

\bigskip
\noindent\emph{Fin de la partie détaillée, Chapitre 8.}