\section{Chapitre 10 – Monte-Carlo global 8D (Conceptuel)}

\subsection{10.1 – Définition du vecteur de paramètres \(\Theta\)}
On considère un vecteur à huit composantes :
\[
  \Theta \;=\;
  \bigl(\alpha_{1},\,T_{c},\,\Delta,\,T_{p},\,\Delta_{p},\,\beta,\,
           \Omega_{m},\,H_{0}\bigr).
\]
Les six premiers paramètres sont propres au MCGT, les deux derniers proviennent de la chaîne MCMC Planck.

\subsection{10.2 – Distributions a priori des paramètres MCGT}
Pour les tirages MCGT, chaque paramètre suit une loi normale centrée sur sa valeur optimale :
\[
  \begin{aligned}
    \alpha_{1} &\sim \mathcal{N}\bigl(0.40,\;0.01^{2}\bigr),
    &\quad
    T_{c} &\sim \mathcal{N}\bigl(0.25\,\text{Gyr},\;0.01^{2}\bigr), \\
    \Delta &\sim \mathcal{N}\bigl(3.00,\;0.50^{2}\bigr),
    &\quad
    T_{p} &\sim \mathcal{N}\bigl(0.05\,\text{Gyr},\;0.01^{2}\bigr), \\
    \Delta_{p} &\sim \mathcal{N}\bigl(0.10\,\text{Gyr},\;0.02^{2}\bigr),
    &\quad
    \beta &\sim \mathcal{N}\bigl(5\times10^{-9},\;(2\times10^{-10})^{2}\bigr).
  \end{aligned}
\]
Un total de \(N_{\rm MC}=10^{5}\) échantillons est généré, avec `np.random.seed(0)` pour assurer la reproductibilité.

\subsection{10.3 – Extraction des paramètres Planck}
Les deux paramètres \(\Omega_{m}\) et \(H_{0}\) sont obtenus à partir d’une chaîne MCMC Planck (\texttt{planck2018\_chain.csv}). Pour chaque échantillon, on calcule :
\[
  \Omega_{m} \;=\; \frac{\omega_{b} h^{2} \;+\;\omega_{c} h^{2}}{h^{2}},
  \quad
  h \;=\; \frac{H_{0}}{100}.
\]
Ainsi, chaque échantillon Planck fournit une paire \(\bigl(\Omega_{m},\,H_{0}\bigr)\).

\subsection{10.4 – Constitution des vecteurs \(\Theta^{(i)}\)}
Pour \(i=1,\dots,N_{\rm MC}\), on combine :
\begin{itemize}
  \item Le sixième tirage MCGT \(\{\alpha_{1}^{(i)},T_{c}^{(i)},\Delta^{(i)},T_{p}^{(i)},\Delta_{p}^{(i)},\beta^{(i)}\}\).
  \item Une paire Planck \(\{\Omega_{m}^{(i)},H_{0}^{(i)}\}\), choisie cycle par cycle parmi les échantillons de la chaîne.
\end{itemize}
On obtient ainsi \(N_{\rm MC}\) vecteurs complets \(\Theta^{(i)} \in \mathbb{R}^{8}\).

\subsection{10.5 – Formule de la covariance empirique}
La matrice de covariance empirique \(C_{8\times8}\) de l’échantillon \(\{\Theta^{(i)}\}\) est définie par :
\[
  C_{jk}
  \;=\;
  \frac{1}{N_{\rm MC} - 1}
  \sum_{i=1}^{N_{\rm MC}}
    \Bigl(\Theta_{j}^{(i)} - \overline{\Theta}_{j}\Bigr)
    \Bigl(\Theta_{k}^{(i)} - \overline{\Theta}_{k}\Bigr),
  \quad
  \overline{\Theta}_{j}
  \;=\;
  \frac{1}{N_{\rm MC}}
  \sum_{i=1}^{N_{\rm MC}} \Theta_{j}^{(i)}.
\]
Cette matrice symétrique 8 × 8 est sauvegardée au format binaire NumPy sous le nom \texttt{10\_global\_covariance\_real.npy}.

\subsection{10.6 – Figures à produire}

\begin{itemize}
  \item \textbf{Heatmap de la matrice de covariance}
    \texttt{fig\_01\_covariance\_heatmap.png} :
    représentation colorée de \(C_{8\times8}\) avec annotations numériques.
    Permet de visualiser corrélations et anticorrélations entre les huit paramètres.

  \item \textbf{Densités KDE unidimensionnelles}
    \texttt{fig\_02\_distributions\_unidimensionnelles.png} :
    pour chaque composante \(\Theta_{j}\), tracer la densité de probabilité estimée (KDE) et indiquer l’écart‐type empirique \(\sqrt{C_{jj}}\).

  \item \textbf{Ellipse de confiance \((\Omega_{m},\,H_{0})\)}
    \texttt{fig\_03\_ellipse\_Omega\_m\_H0.png} :
    ellipses 68 \% (Δχ² = 2,30) et 95 \% (Δχ² = 5,99) basées sur la sous‐matrice covariante extraite de \((\Omega_{m},H_{0})\).
    Le centre correspond à la moyenne Planck \(\bigl(\overline{\Omega}_{m},\,\overline{H}_{0}\bigr)\).

\end{itemize}

\subsection{10.7 – Choix numériques et organisation}

\begin{itemize}
  \item \(\mathbf{N_{\rm MC} = 10^{5}}\) : compromis entre précision statistique et temps de calcul (quelques minutes sur CPU standard).
  \item \(\mathbf{np.random.seed(0)}\) : garantit la même suite de tirages MCGT à chaque exécution.
  \item Ordre des huit paramètres :
    \[
      (\alpha_{1},\,T_{c},\,\Delta,\,T_{p},\,\Delta_{p},\,\beta,\,\Omega_{m},\,H_{0}),
    \]
    strictement conservé dans tous les scripts et figures pour éviter toute confusion.
\end{itemize}

\noindent\emph{(La mise en œuvre détaillée du code Python pour générer la matrice, charger les échantillons Planck, tracer les figures et effectuer les validations est reportée dans le fichier \texttt{10\_covariance\_details.tex}.)}

*****Chapitre 10 (Monte-Carlo global 8D)
Dans la discussion des contraintes combinées (section des résultats globaux), rappelez que sous MCGT, la distribution postérieure de H 0 se décale légèrement vers les valeurs locales grâce à δH0 ∼1%.****

\noindent\emph{Fin du volet conceptuel du Chapitre 10. La partie opérationnelle détaillée commence ci-dessous.}
