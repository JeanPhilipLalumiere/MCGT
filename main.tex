% main.tex — minimal publication document for Chapter 10
\documentclass{article}
\usepackage{graphicx}
\usepackage{hyperref}

\title{MCGT : Modèle de la Courbure Gravitationnelle du Temps}
\author{}
\date{}

\begin{document}
\maketitle

\section*{Chapter 10 -- Monte Carlo Global 8D}

\begin{figure}[ht]
  \centering
  \includegraphics[width=\linewidth]{zz-figures/chapter10/10_fig_01_iso_p95_maps.png}
  \caption{Figure 01: Iso P95 parameter maps.}
\end{figure}

\begin{figure}[ht]
  \centering
  \includegraphics[width=\linewidth]{zz-figures/chapter10/10_fig_02_scatter_phi_at_fpeak.png}
  \caption{Figure 02: Scatter of $\phi$ at $f_{\mathrm{peak}}$.}
\end{figure}

\begin{figure}[ht]
  \centering
  \includegraphics[width=\linewidth]{zz-figures/chapter10/10_fig_03_convergence.png}
  \caption{Figure 03: Convergence diagnostics.}
\end{figure}

\begin{figure}[ht]
  \centering
  \includegraphics[width=\linewidth]{zz-figures/chapter10/10_fig_04_p95_comparison.png}
  \caption{Figure 04: P95 comparison metrics.}
\end{figure}

\begin{figure}[ht]
  \centering
  \includegraphics[width=\linewidth]{zz-figures/chapter10/10_fig_05_hist_cdf_metrics.png}
  \caption{Figure 05: Histogram and CDF metrics.}
\end{figure}

\begin{figure}[ht]
  \centering
  \includegraphics[width=\linewidth]{zz-figures/chapter10/10_fig_06_residual_map.png}
  \caption{Figure 06: Residual map diagnostics.}
\end{figure}

\begin{figure}[ht]
  \centering
  \includegraphics[width=\linewidth]{zz-figures/chapter10/10_fig_07_synthesis.png}
  \caption{Figure 07: Synthesis summary.}
\end{figure}

\begin{figure}[ht]
  \centering
  \includegraphics[width=\linewidth]{zz-figures/chapter10/10_fig_08_diagnostic_histogram.png}
  \caption{Figure 08: Diagnostic histogram.}
\end{figure}

\end{document}
