% main.tex — document principal MCGT
\documentclass\[11pt,a4paper]{book}

% --- Préambule commun (packages, styles, commandes) ---
\input{preambule.tex}

% --- Métadonnées du document ---
\title{Modèle de Courbure Gravitationnelle Temporelle (MCGT)}
\author{Jean-Philip Lalumière}
\date{\today}

\begin{document}
\maketitle

% --- Front matter ---
\frontmatter
\tableofcontents
\listoffigures
\listoftables

% --- Corps principal ---
\mainmatter

% Chapitre 1 – Introduction conceptuelle
\chapter{Introduction conceptuelle}
\input{01-introduction-applications/01\_introduction\_conceptuel}
\input{01-introduction-applications/01\_applications\_calibration\_conceptuel}

% Chapitre 2 – Validation chronologique
\chapter{Validation chronologique}
\input{02-validation-chronologique/02\_validation\_chronologique\_conceptuel}
\input{02-validation-chronologique/02\_validation\_chronologique\_details}

% Chapitre 3 – Stabilité linéaire et extension f(R)
\chapter{Stabilité linéaire et extension f(R)}
\input{03-stabilite-fR/03\_stabilite\_fR\_conceptuel}
\input{03-stabilite-fR/03\_stabilite\_fR\_details}

% Chapitre 4 – Invariants adimensionnels
\chapter{Invariants adimensionnels}
\input{04-invariants-adimensionnels/04\_invariants\_adimensionnels\_conceptuel}
\input{04-invariants-adimensionnels/04\_invariants\_adimensionnels\_details}

% Chapitre 5 – Nucléosynthèse primordiale
\chapter{Nucléosynthèse primordiale}
\input{05-nucleosynthese-primordiale/05\_nucleosynthese\_primordiale\_conceptuel}
\input{05-nucleosynthese-primordiale/05\_nucleosynthese\_primordiale\_details}

% Chapitre 6 – Rayonnement fossile (CMB)
\chapter{Rayonnement fossile (CMB)}
\input{06-rayonnement-cmb/06\_cmb\_conceptuel}
\input{06-rayonnement-cmb/06\_cmb\_details}

% Chapitre 7 – Perturbations scalaires
\chapter{Perturbations scalaires}
\input{07-perturbations-scalaires/07\_perturbations\_scalaires\_conceptuel}
\input{07-perturbations-scalaires/07\_perturbations\_scalaires\_details}

% Chapitre 8 – Couplage sombre modéré
\chapter{Couplage sombre modéré}
\input{08-couplage-sombre/08\_couplage\_sombre\_conceptuel}
\input{08-couplage-sombre/08\_couplage\_sombre\_details}

% Chapitre 9 – Phasage des ondes gravitationnelles
\chapter{Phasage des ondes gravitationnelles}
\input{09-phase-ondes-gravitationnelles/09\_phase\_ondes\_grav\_conceptuel}
\input{09-phase-ondes-gravitationnelles/09\_phase\_ondes\_grav\_details}

% Chapitre 10 – Monte Carlo global 8 dimensions
\chapter{Monte Carlo global 8,dimensions}
\input{10-monte-carlo-global-8d/10\_monte\_carlo\_global\_conceptuel}
\input{10-monte-carlo-global-8d/10\_monte\_carlo\_global\_details}

% --- Annexes (références techniques, données, figures, scripts) ---
\appendix
\chapter{Annexes : Ressources techniques (données, figures, scripts)}
\begin{itemize}
\item Convention et styles (\emph{référence projet}) : \texttt{convention.md}
\item Reproductibilité : \texttt{README-REPRO.md}, procédures : \texttt{RUNBOOK.md}
\item Manifests (publication & diagnostic) : \texttt{zz-manifests/}
\item Schémas de validation JSON/CSV : \texttt{zz-schemas/}
\item Données par chapitre : \texttt{zz-data/chapterX/} (X=01…10)
\item Figures par chapitre : \texttt{zz-figures/chapterX/} (X=01…10)
\item Scripts d’analyse et de tracé : \texttt{zz-scripts/chapterX/} (X=01…10)
\end{itemize}

% --- Back matter ---
\backmatter
\bibliographystyle{unsrt}
\bibliography{references}

\end{document}
