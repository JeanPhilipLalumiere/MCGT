% main.tex — minimal publication document for Chapter 10
\documentclass{article}
\usepackage{graphicx}
\usepackage{hyperref}

\title{MCGT: Modele de la Courbure Gravitationnelle du Temps}
\author{}
\date{}

\begin{document}
\maketitle

\section{Monte Carlo Global 8D}

\begin{figure}[ht]
  \centering
\includegraphics[width=\linewidth]{zz-figures/chapter10/10\_fig\_01\_iso\_p95\_maps.png}
  \caption{Figure 01: Iso P95 parameter maps.}
\end{figure}

\begin{figure}[ht]
  \centering
\includegraphics[width=\linewidth]{zz-figures/chapter10/10\_fig\_02\_scatter\_phi\_at\_fpeak.png}
  \caption{Figure 02: Scatter of $\phi$ at $f_{\mathrm{peak}}$.}
\end{figure}

\begin{figure}[ht]
  \centering
\includegraphics[width=\linewidth]{zz-figures/chapter10/10\_fig\_03\_convergence.png}
  \caption{Figure 03: Convergence diagnostics.}
\end{figure}

\begin{figure}[ht]
  \centering
\includegraphics[width=\linewidth]{zz-figures/chapter10/10\_fig\_04\_p95\_comparison.png}
  \caption{Figure 04: P95 comparison metrics.}
\end{figure}

\begin{figure}[ht]
  \centering
\includegraphics[width=\linewidth]{zz-figures/chapter10/10\_fig\_05\_hist\_cdf\_metrics.png}
  \caption{Figure 05: Histogram and CDF metrics.}
\end{figure}

\begin{figure}[ht]
  \centering
\includegraphics[width=\linewidth]{zz-figures/chapter10/10\_fig\_06\_residual\_map.png}
  \caption{Figure 06: Residual map diagnostics.}
\end{figure}

\begin{figure}[ht]
  \centering
\includegraphics[width=\linewidth]{zz-figures/chapter10/10\_fig\_07\_synthesis.png}
  \caption{Figure 07: Synthesis summary.}
\end{figure}

\begin{figure}[ht]
  \centering
\includegraphics[width=\linewidth]{zz-figures/chapter10/10\_fig\_08\_diagnostic\_histogram\_hist.png}
  \caption{Figure 08: Diagnostic histogram.}
\end{figure}

\end{document}
