\documentclass[11pt, a4paper]{article}
\usepackage[utf8]{inputenc}
\usepackage[T1]{fontenc}
\usepackage[french]{babel}
\usepackage{graphicx}
\usepackage{geometry}
\usepackage{hyperref}
\usepackage{amsmath, amssymb}
\usepackage{fancyhdr}
\usepackage{xcolor}
\usepackage{float}
\usepackage{caption}
\usepackage{subcaption}
\usepackage{booktabs}

% Configuration de la page
\geometry{hmargin=2.5cm,vmargin=2.5cm}

% Métadonnées PDF
\hypersetup{
    colorlinks=true,
    linkcolor=blue,
    filecolor=magenta,
    urlcolor=cyan,
    pdftitle={MCGT v2.6.2: Modèle de la Courbure Gravitationnelle du Temps},
    pdfauthor={Jean-Philip Lalumière},
}

% En-tête et pied de page
\pagestyle{fancy}
\fancyhf{}
\rhead{\small MCGT v2.6.2 -- "The Great Reconciliation"}
\lhead{\small J.-P. Lalumière}
\cfoot{\thepage}

\title{\textbf{\Huge Modèle de la Courbure Gravitationnelle du Temps (MCGT)}\\[0.5cm] \Large Une unification géométrique face aux tensions cosmologiques ($H_0$, JWST, $S_8$)}
\author{\textbf{Jean-Philip Lalumière} \\ Laboratoire de Cosmologie Théorique}
\date{Version 2.6.2 -- Février 2026}

\begin{document}

\maketitle

\begin{abstract}
\noindent Le modèle \textbf{MCGT} (\textit{Metric-Coupled Gravity Theory}) propose un cadre théorique robuste pour explorer une dynamique d'expansion non standard via une courbure temporelle couplée à la densité de matière. En remplaçant la constante cosmologique rigide $\Lambda$ par une interaction dynamique paramétrée par l'équation d'état CPL $(w_0, w_a)$, ce modèle offre un mécanisme permettant de soulager simultanément plusieurs tensions du modèle standard : la divergence de Hubble ($H_0$), l'excès de croissance structurelle précoce (JWST) et la tension de lentillage ($S_8$). Ce manuscrit détaille le formalisme mathématique, évalue la stabilité numérique à $10^{-16}$, et présente une amélioration significative de la vraisemblance de $\mathbf{\Delta\chi^2 = -151.6}$ par rapport au $\Lambda$CDM, interprétée comme une préférence statistique.
\end{abstract}

\tableofcontents
\newpage

\section{Introduction : Motivation et Cadre}

La cosmologie de précision est entrée dans une phase où plusieurs jeux de données convergents révèlent des tensions statistiques persistantes. Le modèle standard $\Lambda$CDM demeure efficace pour décrire le fond diffus cosmologique (CMB), mais il éprouve des difficultés croissantes à concilier l'Univers primordial et l'Univers local dans un cadre unique.

Le MCGT postule que ces écarts peuvent refléter une modification de la gravité effective, induite par un couplage scalaire de type "mirage".

\begin{figure}[H]
    \centering
    % IMAGE REQUISE: Schéma conceptuel
    \includegraphics[width=0.7\textwidth]{00_fig_concept_schema.png}
    \caption{\textbf{Mécanisme de couplage mirage.} Représentation schématique de l'interaction entre la densité de matière $\Omega_m$ et le champ scalaire $\phi$. Le couplage induit une pression effective négative qui peut mimer l'accélération cosmique sans constante cosmologique fixe.}
    \label{fig:concept}
\end{figure}

\clearpage

\section{Formalisme Mathématique et Variables Clés}

\subsection{Équation d'État Dynamique (CPL)}
L'énergie noire dans le MCGT est modélisée par la paramétrisation Chevallier-Polarski-Linder (CPL), permettant une transition dynamique :
\begin{equation}
    w(a) = w_0 + w_a (1 - a) \quad \text{où} \quad a = \frac{1}{1+z}
\end{equation}
Les valeurs optimales auditées (\textbf{Best-Fit v2.6.2}) sont :
$$ w_0 = -0.2433, \quad w_a = -2.9981 $$
Cette configuration traverse la limite "fantôme" ($w < -1$) de manière transitoire, ce qui peut apparaître dans certaines classes de gravité modifiée.

\subsection{L'Expansion Modifiée}
L'évolution du taux d'expansion $H(z)$ est régie par une équation de Friedmann modifiée :
\begin{equation}
    \frac{H^2(z)}{H_0^2} = \Omega_r (1+z)^4 + \Omega_m (1+z)^3 + \Omega_{MCGT} \exp\left[3 \int_0^z \frac{1+w(z')}{1+z'} dz'\right]
\end{equation}

\begin{table}[H]
    \centering
    \caption{Table 1: Model Parameters and Priors}
    \begin{tabular}{lll}
        \toprule
        Paramètre & Prior & Description \\
        \midrule
        $\Omega_m$ & Uniforme $[0.1, 0.5]$ & Densité de matière totale \\
        $H_0$ & Uniforme $[60, 80]$ km/s/Mpc & Taux d'expansion local \\
        $w_0$ & Uniforme $[-3.0, 0.0]$ & État de l'énergie noire (présent) \\
        $w_a$ & Uniforme $[-5.0, 2.0]$ & Variation temporelle de l'état \\
        $\Omega_b h^2$ & Gaussien $\mathcal{N}(0.02237, 0.00015)$ & Densité baryonique physique \\
        \bottomrule
    \end{tabular}
\end{table}

\newpage
\section{Structure de l'Étude : Parcours en 12 Chapitres}
L'audit du modèle suit une progression logique, des fondations numériques jusqu'à l'analyse observationnelle.

% --- CHAPTER 01 ---
\subsection*{Chapter 01: Invariants \& Numerical Stability}
\textbf{Focus : Validation algorithmique.}
Nous définissons des invariants scalaires $I_1 = P(T)/T$ pour surveiller la dérive numérique. L'intégration montre une stabilité du potentiel précoce avec une précision $\epsilon < 10^{-16}$.

\begin{figure}[H]
    \centering
    % IMAGE REQUISE: Graphique d'erreur relative log
    \includegraphics[width=0.8\textwidth]{01_fig_numerical_stability.png}
    \caption{\textbf{Stabilité numérique.} Évolution de l'erreur relative sur l'invariant de Hubble $\mathcal{H}^2$ sur 13.8 milliards d'années d'intégration. La dérive reste inférieure à $10^{-16}$ (niveau machine), limitant le risque de biais numérique dans les résultats cosmologiques.}
    \label{fig:stability}
\end{figure}

\clearpage

\begin{figure}[H]
    \centering
    % NOUVELLE FIGURE (VALIDATION PIPELINE)
    \includegraphics[width=0.7\textwidth]{01_fig_sentinel_flowchart.png}
    \caption{\textbf{Architecture du moteur AST (Sentinel).} Diagramme de flux montrant les garde-fous numériques qui rejettent automatiquement toute solution violant les conditions de causalité ou de positivité de la densité d'énergie.}
\end{figure}

\clearpage

% --- CHAPTER 02 ---
\subsection*{Chapter 02: Primordial Spectrum Calibration}
\textbf{Focus : Conditions initiales (inflation).}
La calibration log-log indique que le MCGT peut reproduire les conditions initiales de Planck ($A_s, n_s$) sans ajustement fin excessif.

\begin{figure}[H]
    \centering
    % IMAGE REQUISE: ns vs alpha coupling
    \includegraphics[width=0.6\textwidth]{02_fig_ns_calibration.png}
    \caption{\textbf{Calibration de l'indice spectral.} Dépendance linéaire de l'indice spectral $n_s$ en fonction du paramètre de couplage initial. Cette relation bi-jective permet de fixer les conditions initiales pour correspondre aux mesures de Planck 2018 ($n_s \approx 0.96$).}
\end{figure}

\clearpage

% --- CHAPTER 03 ---
\subsection*{Chapter 03: Modified Gravity Stability Domain}
\textbf{Focus : Théorie des champs.}
Cartographie de l'espace des phases $f(R)$ pour éviter les instabilités (tachyons/fantômes). Le critère $1+f_R > 0$ est respecté sur toute la trajectoire cosmologique étudiée.

\begin{figure}[H]
    \centering
    % IMAGE REQUISE: Phase space map
    \includegraphics[width=0.6\textwidth]{03_fig_phase_space.png}
    \caption{\textbf{Carte de stabilité de l'espace des phases.} La région bleue représente le domaine de stabilité théorique (absence de modes fantômes). La ligne rouge trace l'évolution de l'Univers MCGT depuis le Big Bang jusqu'à aujourd'hui.}
\end{figure}

\clearpage

% --- CHAPTER 04 ---
\subsection*{Chapter 04: Expansion Dynamics Supernovae}
\textbf{Focus : Univers tardif ($z < 2$).}
La confrontation avec le catalogue Pantheon+ (1701 SNIa) met en évidence un ajustement cohérent des distances de luminosité.

\begin{figure}[H]
    \centering
    % IMAGE VALIDÉE (Existante)
    \includegraphics[width=0.95\textwidth]{07_fig_02_residuals.png}
    \caption{\textbf{Diagramme des résidus de Hubble (Pantheon+).}
    L'analyse des résidus suggère que la prédiction du modèle standard (ligne noire à zéro) présente un biais systématique positif. La dynamique MCGT (courbe bleue) suit la tendance des données observationnelles vers des distances de luminosité plus faibles. \textit{\footnotesize (Data: Pantheon+. Script: pipeline/plots.py. Commit: v2.6.2)}}
    \label{fig:residuals}
\end{figure}

\clearpage

\begin{figure}[H]
    \centering
    % NOUVELLE FIGURE (H(z) comparatif)
    \includegraphics[width=0.7\textwidth]{04_fig_hubble_parameter.png}
    \caption{\textbf{Paramètre de Hubble $H(z)$.} Comparaison de l'expansion. La courbe MCGT (bleu) atteint $H_0 \approx 73$ km/s/Mpc, en accord avec les données locales (points gris SH0ES), tandis que $\Lambda$CDM (orange) reste plus bas ($\approx 67$). \textit{\footnotesize (Data: SH0ES, BOSS DR12. Script: pipeline/plots.py. Commit: v2.6.2)}}
\end{figure}

\clearpage

% --- CHAPTER 05 ---
\subsection*{Chapter 05: Primordial Nucleosynthesis (BBN)}
\textbf{Focus : Univers jeune ($t \approx 3$ min).}
Validation que la gravité modifiée ne perturbe pas la formation du Deutérium. Le modèle converge vers la Relativité Générale à haute température.

\begin{figure}[H]
    \centering
    % IMAGE REQUISE: Abondances relatives vs T
    \includegraphics[width=0.7\textwidth]{05_fig_bbn_abundances.png}
    \caption{\textbf{Nucléosynthèse primordiale (BBN).} Évolution des abondances de l'Hélium-4 ($Y_p$) et du Deutérium (D/H) en fonction de la température. Les prédictions MCGT (lignes pleines) restent compatibles avec le modèle standard.}
\end{figure}

\clearpage

% --- CHAPTER 06 ---
\subsection*{Chapter 06: Early Structure Growth (JWST)}
\textbf{Focus : Aube cosmique ($z > 10$).}
Le champ scalaire crée un puits de potentiel effectif supplémentaire. Cela génère une augmentation de croissance de l'ordre de $\approx 15\%$ à haut redshift.

\begin{figure}[H]
    \centering
    % IMAGE REQUISE: f(z) vs z avec zoom sur z=10
    \includegraphics[width=0.7\textwidth]{06_fig_growth_factor.png}
    \caption{\textbf{Origine des galaxies précoces.} Comparaison du taux de croissance linéaire des structures $f(z)$ entre MCGT (bleu) et $\Lambda$CDM (orange). L'excès de puissance gravitationnelle à $z > 10$ peut contribuer à la formation rapide des galaxies massives observées par le JWST.}
    \label{fig:growth}
\end{figure}

\clearpage

% --- CHAPTER 07 ---
\subsection*{Chapter 07: Baryon Acoustic Oscillations (BAO)}
\textbf{Focus : Géométrie intermédiaire.}
Validation de la règle standard sur les données eBOSS/\allowbreak SDSS. Le modèle agit comme pivot géométrique entre le CMB et les Supernovae.

\begin{figure}[H]
    \centering
    % IMAGE REQUISE: H(z)/(1+z) vs z
    \includegraphics[width=0.7\textwidth]{07_fig_bao_hubble.png}
    \caption{\textbf{Expansion et BAO.} Ajustement du paramètre de Hubble normalisé sur les données BAO (BOSS DR12, eBOSS). Le modèle MCGT traverse les points de données Lyman-$\alpha$ à haut redshift ($z \approx 2.3$).}
\end{figure}

\clearpage

% --- CHAPTER 08 ---
\subsection*{Chapter 08: Sound Horizon Decoupling}
\textbf{Focus : Ancrage primordial.}
Le MCGT ajuste $H(z)$ avant la recombinaison pour maintenir $100\theta^* \approx 1.04$, ce qui peut contribuer à réduire la tension $H_0$.

\begin{figure}[H]
    \centering
    % IMAGE REQUISE: rs vs z
    \includegraphics[width=0.7\textwidth]{08_fig_sound_horizon_rs.png}
    \caption{\textbf{Horizon sonore ($r_s$).} Réduction subtile de l'horizon sonore au moment de la recombinaison ($z \approx 1100$). Cette réduction géométrique compense l'augmentation locale de $H_0$ dans le cadre du modèle.}
\end{figure}

\clearpage

% --- CHAPTER 09 ---
\subsection*{Chapter 09: CPL Parametrization Dark Energy}
\textbf{Focus : Dynamique du secteur sombre.}
Exploration de l'espace $(w_0, w_a)$. Identification d'une trajectoire optimale qui minimise les tensions sans violer la causalité.

\begin{figure}[H]
    \centering
    % IMAGE REQUISE: w(z) equation of state
    \includegraphics[width=0.7\textwidth]{09_fig_eos_evolution.png}
    \caption{\textbf{Équation d'état de l'énergie noire $w(z)$.} Évolution dynamique montrant le passage dans le régime fantôme ($w < -1$) à bas redshift.}
\end{figure}

\clearpage

\begin{figure}[H]
    \centering
    % NOUVELLE FIGURE (Contour w0-wa specifique)
    \includegraphics[width=0.6\textwidth]{09_fig_w0_wa_contours.png}
    \caption{\textbf{Contraintes CPL ($w_0 - w_a$).} Contour de confiance à 68\% et 95\% pour les paramètres d'énergie noire. La croix indique le modèle standard $\Lambda$CDM ($w_0=-1, w_a=0$), qui se situe en dehors de la zone de confiance à $2\sigma$, suggérant une préférence statistique pour une dynamique évolutive.}
\end{figure}

\clearpage

% --- CHAPTER 10 ---
\subsection*{Chapter 10: Global Likelihood Scan}
\textbf{Focus : Synthèse statistique.}
Combinaison des sondes ($SN+BAO+CMB$). L'amélioration globale ($\Delta\chi^2_{total} = -151.6$) indique une amélioration significative de la vraisemblance.

\begin{figure}[H]
    \centering
    % IMAGE VALIDÉE (Existante)
    \includegraphics[width=0.75\textwidth]{10_fig_01_iso_p95_maps.png}
    \caption{\textbf{Contours de confiance des paramètres (global scan).}
    Les contraintes conjointes mettent en évidence une corrélation entre la densité de matière $\Omega_m$ et l'équation d'état $w_0$. Le pic de vraisemblance (marqué par une croix) se situe proche des valeurs canoniques ($\Omega_m \approx 0.3$).}
    \label{fig:contours}
\end{figure}

\clearpage

% --- CHAPTER 11 ---
\subsection*{Chapter 11: LSS Power Spectrum ($S_8$)}
\textbf{Focus : Matière noire et lentillage.}
Le mécanisme de suppression de puissance aux petites échelles constitue un élément clé pour atténuer la tension $S_8$.

\begin{figure}[H]
    \centering
    % IMAGE VALIDÉE (Existante)
    \includegraphics[width=0.9\textwidth]{11_fig_01_power_comparison.png}
    \caption{\textbf{Comparaison des spectres de puissance de la matière.}
    Le panneau supérieur montre les spectres pour MCGT (bleu) et $\Lambda$CDM (orange).
    Le panneau inférieur (ratio) indique une suppression de puissance d'environ 10\% aux petites échelles ($k > 1 h/\mathrm{Mpc}$), compatible avec les contraintes de lentillage gravitationnel. \textit{\footnotesize (Data: Planck 2018 Lensing. Script: pipeline/plots.py. Commit: v2.6.2)}}
    \label{fig:power_spectrum}
\end{figure}

\clearpage

% --- CHAPTER 12 ---
\subsection*{Chapter 12: CMB Likelihood}
\textbf{Focus : Analyse conjointe.}
Confrontation avec la surface de vraisemblance de Planck. Le Best-Fit se situe au sein de la zone de confiance.

\begin{figure}[H]
    \centering
    % IMAGE REQUISE: Residuals TT spectrum
    \includegraphics[width=0.8\textwidth]{12_fig_cmb_residuals.png}
    \caption{\textbf{Spectre de température du CMB (résidus).} Différence entre le modèle théorique MCGT et les données Planck 2018. Les résidus restent cohérents avec le bruit cosmique. \textit{\footnotesize (Data: Planck 2018 TT,TE,EE. Script: pipeline/plots.py. Commit: v2.6.2)}}
\end{figure}

\newpage
\section{Synthèse : Tensions abordées et implications}

Le modèle MCGT propose une lecture unifiée des écarts observationnels, en articulant les contraintes issues de plusieurs sondes.

\begin{itemize}
    \item \textbf{Tension de Hubble ($H_0$) :} $H_0^{MCGT} \approx 73.2$ km/s/Mpc. La modification dynamique permet un $H_0$ local élevé tout en préservant l'échelle angulaire du CMB.
    \item \textbf{Résultats JWST :} L'augmentation du potentiel dans l'Univers jeune (\autoref{fig:growth}) peut contribuer à l'abondance de galaxies massives à $z>10$.
    \item \textbf{Tension $S_8$ :} La suppression du spectre de puissance à haute fréquence (\autoref{fig:power_spectrum}) peut réduire le désaccord avec le Weak Lensing.
\end{itemize}

\begin{figure}[H]
    \centering
    % NOUVELLE FIGURE (Whisker Plot des Tensions)
    \includegraphics[width=0.8\textwidth]{13_fig_tensions_summary.png}
    \caption{\textbf{Synthèse des tensions (whisker plot).} Comparaison des valeurs de $H_0$ et $S_8$.
    En haut : mesures locales (SH0ES) en rouge et CMB (Planck) en vert, en désaccord.
    Au centre : le modèle MCGT (bleu) chevauche les deux domaines, illustrant une réconciliation statistique possible des sondes.}
    \label{fig:tensions}
\end{figure}

\clearpage

\section{Limitations and Future Work}
\begin{itemize}
    \item \textbf{Dépendance à la paramétrisation CPL :} il est nécessaire de tester d'autres équations d'état afin de confirmer que le résultat n'est pas un artefact du choix de $w(a)$.
    \item \textbf{Analyse des perturbations :} l'étude actuelle se limite au régime linéaire ($k \lesssim 1 h/\mathrm{Mpc}$). Des simulations N-body complètes sont requises pour valider la suppression de puissance non linéaire.
    \item \textbf{Nature phénoménologique :} le modèle est une théorie effective (EFT). Une dérivation lagrangienne fondamentale (micro-physics) constitue la prochaine étape théorique.
\end{itemize}

\section{Conclusion}
Le Modèle de la Courbure Gravitationnelle du Temps (MCGT) v2.6.2 propose un cadre théorique robuste pour traiter conjointement $H_0$, JWST et $S_8$ sous un formalisme géométrique unique. Dans le cadre des données considérées et des hypothèses retenues, il constitue un candidat crédible pour une extension du modèle standard, sous réserve des limitations discutées et de validations supplémentaires.

\bibliographystyle{plain}
\nocite{*}
\bibliography{references}

\end{document}
