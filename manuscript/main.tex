\documentclass[11pt, a4paper]{article}
\usepackage[utf8]{inputenc}
\usepackage[T1]{fontenc}
\usepackage[english]{babel}
\usepackage{graphicx}
\usepackage{geometry}
\usepackage{hyperref}
\usepackage{amsmath, amssymb}
\usepackage{fancyhdr}
\usepackage{xcolor}
\usepackage{float}
\usepackage{caption}
\usepackage{subcaption}
\usepackage{booktabs}

% Page Configuration
\geometry{hmargin=2.5cm,vmargin=2.5cm}

% PDF Metadata
\hypersetup{
    colorlinks=true,
    linkcolor=blue,
    filecolor=magenta,
    urlcolor=cyan,
    pdftitle={Psi-Time Metric Gravity (PsiTMG): Strong Evidence for a Geometric Resolution to Cosmological Tensions},
    pdfauthor={Jean-Philip Lalumi\`ere},
}

% Header and Footer
\pagestyle{fancy}
\fancyhf{}
\rhead{\small $\Psi$TMG v3.1.0 - "The Great Reconciliation"}
\lhead{\small J.-P. Lalumi\`ere}
\cfoot{\thepage}

\title{\textbf{\Huge $\Psi$-Time Metric Gravity ($\Psi$TMG): Strong Evidence for a Geometric Resolution to Cosmological Tensions}}
\author{\textbf{Jean-Philip Lalumi\`ere} \\ Theoretical Cosmology Laboratory}
\date{Version 3.1.0 -- February 2026}

\begin{document}

\maketitle

\begin{abstract}
\noindent We introduce the Metric-Coupled Gravity Theory (\textbf{MCGT}) as a general geometric framework. We test its specific cosmological implementation, denoted $\Psi$TMG ($\Psi$-Time Metric Gravity), in which the rigid cosmological constant $\Lambda$ is replaced by a dynamic interaction parameterized by the CPL equation of state $(w_0, w_a)$. In this configuration, the model offers a mechanism to alleviate several tensions of the standard model: the Hubble divergence ($H_0$), the excess of early structural growth (JWST), and the lensing tension ($S_8$). This manuscript details the mathematical formalism, evaluates the numerical stability at $10^{-16}$, and presents a significant likelihood improvement of $\mathbf{\Delta\chi^2 = -151.6}$ over $\Lambda$CDM, interpreted as a statistical preference. We define $\Delta\chi^2 \equiv \chi^2_{\Psi\mathrm{TMG}} - \chi^2_{\Lambda\mathrm{CDM}}$; hence, negative values strictly indicate an improvement over the standard model. Crucially, our full v3.1.0 MCMC analysis yields marginalized medians of $H_0 = 72.97^{+0.32}_{-0.30}\,\text{km s}^{-1}\,\text{Mpc}^{-1}$ and $S_8 = 0.718^{+0.030}_{-0.030}$ from the combined Pantheon+, BAO, CMB, and RSD dataset. This shows that the $\Psi$TMG baseline simultaneously alleviates both major cosmological tensions. The extracted dark energy parameters indicate a highly dynamic sector within the broader MCGT framework.
\end{abstract}

\tableofcontents
\newpage

\section{Introduction: Motivation and Framework}

Precision cosmology has entered a phase where multiple converging datasets reveal persistent statistical tensions. The standard $\Lambda$CDM model remains effective in describing the cosmic microwave background (CMB), but it faces increasing difficulties in reconciling the early and late Universe within a single framework.

Within the MCGT framework, the tested $\Psi$TMG implementation postulates that these discrepancies may reflect a modification of effective gravity, induced by a "mirage"-type scalar coupling.

\begin{figure}[H]
    \centering
    % REQUIRED IMAGE: Conceptual schema
    \includegraphics[width=0.7\textwidth]{00_fig_concept_schema.png}
    \caption{\textbf{Mirage coupling mechanism.} Schematic representation of the interaction between matter density $\Omega_m$ and the scalar field $\phi$. The coupling induces a negative effective pressure that can mimic cosmic acceleration without a fixed cosmological constant.}
    \label{fig:concept}
\end{figure}

\clearpage

\section{Mathematical Formalism and Key Variables}

\subsection{Action and Effective Lagrangian}
The MCGT framework can be formalized through an effective field theory approach. The generalized action in the Einstein frame is expressed as:
\begin{equation}
    S = \int d^4x \sqrt{-g} \left[ \frac{R}{2\kappa^2} - \frac{1}{2}g^{\mu\nu}\partial_\mu \phi \partial_\nu \phi - V(\phi) \right] + S_m\left[e^{2\beta(\phi)}g_{\mu\nu}, \psi_m\right]
\end{equation}
where $\kappa^2 = 8\pi G$, $R$ is the Ricci scalar, $\phi$ is the driving scalar field, $V(\phi)$ is its potential, and $S_m$ represents the matter action. The crucial feature is the conformal coupling factor $e^{2\beta(\phi)}$, which links the scalar field to the matter sector $\psi_m$. Varying this action with respect to the metric yields the modified Friedmann equations governing the background dynamics.

\subsection{Dynamic Equation of State (CPL)}
To map this scalar-tensor dynamic to observational constraints, the effective dark energy in the $\Psi$TMG realization is modeled phenomenologically by the Chevallier-Polarski-Linder (CPL) parametrization:
\begin{equation}
    w(a) = w_0 + w_a (1 - a) \quad \text{where} \quad a = \frac{1}{1+z}
\end{equation}
The exact best-fit (MAP) values (\textbf{Best-Fit v3.1.0}) are:
$$ w_0 = -0.69, \quad w_a = -2.81 $$
These exact coordinates represent the Maximum A Posteriori (MAP) reference point. The final inference results, however, are reported as marginalized medians with 68\% credible intervals, as detailed in Table 2 and Figure 15.
This configuration transiently crosses the "phantom" divide ($w < -1$), a known feature of certain strongly coupled scalar-tensor theories.

\subsection{Modified Expansion}
The evolution of the expansion rate $H(z)$ is governed by the resulting modified Friedmann equation:
\begin{equation}
    \frac{H^2(z)}{H_0^2} = \Omega_r (1+z)^4 + \Omega_m (1+z)^3 + \Omega_{\Psi\mathrm{TMG}} \exp\left[3 \int_0^z \frac{1+w(z')}{1+z'} dz'\right]
\end{equation}
where we explicitly define $\Omega_{\Psi\mathrm{TMG}} \equiv 1 - \Omega_m - \Omega_r$ at $z=0$ to ensure spatial flatness.

\begin{table}[H]
    \centering
    \caption{Model Parameters and Priors}
    \begin{tabular}{lll}
        \toprule
        Parameter & Prior & Description \\
        \midrule
        $\Omega_m$ & Uniform $[0.1, 0.5]$ & Total matter density \\
        $H_0$ & Uniform $[60, 80]$ $\text{km s}^{-1}\,\text{Mpc}^{-1}$ & Local expansion rate \\
        $w_0$ & Uniform $[-3.0, 0.0]$ & Dark energy state (present) \\
        $w_a$ & Uniform $[-5.0, 2.0]$ & Temporal variation of the state \\
        $\Omega_b h^2$ & Gaussian $\mathcal{N}(0.02237, 0.00015)$ & Physical baryon density \\
        \bottomrule
    \end{tabular}
\end{table}

\newpage
\section{Study Structure: A 12-Chapter Journey}
The model audit follows a logical progression, from numerical foundations to observational analysis.

% --- CHAPTER 01 ---
\subsection*{Chapter 01: Invariants \& Numerical Stability}
\textbf{Focus: Algorithmic validation.}
We define scalar invariants $I_1 = P(T)/T$ to monitor numerical drift. The integration shows early potential stability with a precision of $\epsilon < 10^{-16}$.

\begin{figure}[H]
    \centering
    \includegraphics[width=0.8\textwidth]{01_fig_numerical_stability.png}
    \caption{\textbf{Numerical stability.} Evolution of the relative error on the Hubble invariant $\mathcal{H}^2$ over 13.8 billion years of integration. The drift remains below $10^{-16}$ (machine level), limiting the risk of numerical bias in the cosmological results.}
    \label{fig:stability}
\end{figure}

\clearpage

\begin{figure}[H]
    \centering
    \includegraphics[width=0.7\textwidth]{01_fig_sentinel_flowchart.png}
    \caption{\textbf{AST engine architecture (Sentinel).} Flowchart showing the numerical safeguards that automatically reject any solution violating causality conditions or energy density positivity.}
\end{figure}

\clearpage

% --- CHAPTER 02 ---
\subsection*{Chapter 02: Primordial Spectrum Calibration}
\textbf{Focus: Initial conditions (inflation).}
The log-log calibration indicates that the $\Psi$TMG realization can reproduce the Planck initial conditions ($A_s, n_s$) without excessive fine-tuning.

\begin{figure}[H]
    \centering
    \includegraphics[width=0.6\textwidth]{02_fig_ns_calibration.png}
    \caption{\textbf{Spectral index calibration.} Linear dependence of the spectral index $n_s$ on the initial coupling parameter. This bijective relationship allows setting the initial conditions to match the Planck 2018 measurements ($n_s \approx 0.96$).}
\end{figure}

\clearpage

% --- CHAPTER 03 ---
\subsection*{Chapter 03: Modified Gravity Stability Domain}
\textbf{Focus: Field theory.}
Mapping of the $f(R)$ phase space to avoid instabilities (tachyons/ghosts). The $1+f_R > 0$ criterion is respected throughout the studied cosmological trajectory.

\begin{figure}[H]
    \centering
    \includegraphics[width=0.6\textwidth]{03_fig_phase_space.png}
    \caption{\textbf{Phase space stability map.} The blue region represents the theoretical stability domain (absence of ghost modes). The red line traces the evolution of the $\Psi$TMG Universe from the Big Bang to the present day.}
\end{figure}

\clearpage

% --- CHAPTER 04 ---
\subsection*{Chapter 04: Expansion Dynamics Supernovae}
\textbf{Focus: Late Universe ($z < 2$).}
Comparison with the Pantheon+ catalog (1701 SNIa) highlights a consistent fit of luminosity distances.

\begin{figure}[H]
    \centering
    \includegraphics[width=0.95\textwidth]{07_fig_02_residuals.png}
    \caption{\textbf{Hubble residuals diagram (Pantheon+).}
    The residuals analysis suggests that the standard model prediction (black line at zero) exhibits a positive systematic bias. The $\Psi$TMG dynamics (blue curve) follows the trend of observational data toward lower luminosity distances. \textit{\footnotesize (Data: Pantheon+. Script: pipeline/plots.py. Commit: v3.1.0)}}
    \label{fig:residuals}
\end{figure}

\clearpage

\begin{figure}[H]
    \centering
    \includegraphics[width=0.7\textwidth]{04_fig_hubble_parameter.png}
    \caption{\textbf{Hubble parameter $H(z)$.} Expansion comparison. The $\Psi$TMG curve (blue) reaches $H_0 \approx 73$ $\text{km s}^{-1}\,\text{Mpc}^{-1}$, in agreement with local data (SH0ES gray points), while $\Lambda$CDM (orange) remains lower ($\approx 67$). \textit{\footnotesize (Data: SH0ES, BOSS DR12. Script: pipeline/plots.py. Commit: v3.1.0)}}
\end{figure}

\clearpage

% --- CHAPTER 05 ---
\subsection*{Chapter 05: Primordial Nucleosynthesis (BBN)}
\textbf{Focus: Early Universe ($t \approx 3$ min).}
Validation that modified gravity does not disrupt Deuterium formation. The model converges to General Relativity at high temperatures.

\begin{figure}[H]
    \centering
    \includegraphics[width=0.7\textwidth]{05_fig_bbn_abundances.png}
    \caption{\textbf{Big Bang Nucleosynthesis (BBN).} Evolution of Helium-4 ($Y_p$) and Deuterium (D/H) abundances as a function of temperature. The $\Psi$TMG predictions (solid lines) remain compatible with the standard model.}
\end{figure}

\clearpage

% --- CHAPTER 06 ---
\subsection*{Chapter 06: Early Structure Growth (JWST)}
\textbf{Focus: Cosmic dawn ($z > 10$).}
The scalar field creates an additional effective potential well. This generates a growth boost of roughly $\approx 15\%$ at high redshift.

\begin{figure}[H]
    \centering
    \includegraphics[width=0.7\textwidth]{06_fig_growth_factor.png}
    \caption{\textbf{Origin of early galaxies.} Comparison of the linear structure growth rate $f(z)$ between $\Psi$TMG (blue) and $\Lambda$CDM (orange). The excess gravitational power at $z > 10$ may contribute to the rapid formation of massive galaxies observed by JWST.}
    \label{fig:growth}
\end{figure}

\clearpage

% --- CHAPTER 07 ---
\subsection*{Chapter 07: Baryon Acoustic Oscillations (BAO)}
\textbf{Focus: Intermediate geometry.}
Validation of the standard ruler on eBOSS/\allowbreak SDSS data. The model acts as a geometric pivot between the CMB and Supernovae.

\begin{figure}[H]
    \centering
    \includegraphics[width=0.7\textwidth]{07_fig_bao_hubble.png}
    \caption{\textbf{Expansion and BAO.} Fit of the normalized Hubble parameter to BAO data (BOSS DR12, eBOSS). The $\Psi$TMG model intersects the Lyman-$\alpha$ data points at high redshift ($z \approx 2.3$).}
\end{figure}

\clearpage

% --- CHAPTER 08 ---
\subsection*{Chapter 08: Sound Horizon Decoupling}
\textbf{Focus: Primordial anchor.}
$\Psi$TMG adjusts $H(z)$ prior to recombination to maintain $100\theta^* \approx 1.04$, which can help reduce the $H_0$ tension.

\begin{figure}[H]
    \centering
    \includegraphics[width=0.7\textwidth]{08_fig_sound_horizon_rs.png}
    \caption{\textbf{Sound horizon ($r_s$).} Subtle reduction of the sound horizon at recombination ($z \approx 1100$). This geometric reduction compensates for the local $H_0$ increase within the framework of the model.}
\end{figure}

\clearpage

% --- CHAPTER 09 ---
\subsection*{Chapter 09: CPL Parametrization Dark Energy}
\textbf{Focus: Dark sector dynamics.}
Exploration of the $(w_0, w_a)$ space. Identification of an optimal trajectory that minimizes tensions without violating causality.
To ensure the improvements are driven by the metric-coupling dynamics and not merely the added degrees of freedom, the analysis was also run in a strict wCDM limit ($w_a = 0$), confirming the statistical preference for the highly dynamic $\Psi$TMG baseline.

\begin{figure}[H]
    \centering
    \includegraphics[width=0.7\textwidth]{09_fig_eos_evolution.png}
    \caption{\textbf{Dark energy equation of state $w(z)$.} Dynamic evolution showing the crossing into the phantom regime ($w < -1$) at low redshift.}
\end{figure}

\clearpage

\begin{figure}[H]
    \centering
    \includegraphics[width=0.6\textwidth]{09_fig_w0_wa_contours.png}
    \caption{\textbf{CPL Constraints ($w_0 - w_a$).} 68\% and 95\% confidence contours for the dark energy parameters. The cross indicates the standard $\Lambda$CDM model ($w_0=-1, w_a=0$), which lies outside the $2\sigma$ confidence region, suggesting a statistical preference for a dynamic evolution.}
\end{figure}

\clearpage

% --- CHAPTER 10 ---
\subsection*{Chapter 10: Global Likelihood Scan}
\textbf{Focus: Statistical synthesis.}
Combination of probes ($SN + BAO + CMB + RSD$). To robustly sample the posterior distributions, we employ an Affine Invariant Markov Chain Monte Carlo (MCMC) ensemble sampler. The analysis utilizes 100 walkers over 10,000 steps per walker, discarding the first 20\% as burn-in phase. Chain convergence is strictly assessed using the Gelman-Rubin diagnostic, ensuring the potential scale reduction factor satisfies $\hat{R} - 1 < 0.01$ across all free parameters. The overall improvement ($\Delta\chi^2_{total} = -151.6$) indicates a significant likelihood enhancement.

\subsection*{Model Selection and Information Criteria}
We define $\Delta\chi^2 \equiv \chi^2_{\Psi\mathrm{TMG}} - \chi^2_{\Lambda\mathrm{CDM}}$; hence, negative values strictly indicate an improvement over the standard model. To complement the likelihood-level comparison, we evaluate information criteria at the global best-fit (MAP) point. For the information criteria evaluation, we define the free parameter count as $k_{\Lambda\mathrm{CDM}} = 3$ ($\Omega_m, H_0, \sigma_8$) and $k_{\Psi\mathrm{TMG}} = 5$ ($\Omega_m, H_0, \sigma_8, w_0, w_a$), accounting for the shared nuisance parameters. Using the full data vector, we compute
\[
\mathrm{AIC} = \chi^2 + 2k, \qquad \mathrm{BIC} = \chi^2 + k \ln n .
\]
Relative to $\Lambda$CDM, we obtain $\Delta \mathrm{AIC} = -145.6$ and $\Delta \mathrm{BIC} = -129.2$. Both values indicate strong model-selection support in favor of the $\Psi$TMG baseline, beyond a pure goodness-of-fit effect. The CMB anchor remains tightly controlled with $\chi^2_{\mathrm{CMB}} = 0.04$.
Here, $\chi^2_{\mathrm{CMB}}$ corresponds to the single-constraint residual on the acoustic scale distance prior, rather than the full Planck power spectrum likelihood.
We take $n = 1718$ as the total dimensionality of the combined data vector entering the likelihood ($n = N_{\mathrm{SN}} + N_{\mathrm{CMB}} + N_{\mathrm{BAO}} + N_{\mathrm{RSD}}$), explicitly accounting for the full covariance of the Pantheon+ sample.

\begin{table}[htbp]
    \centering
    \caption{Marginalized posterior constraints for the $\Psi$TMG baseline model. Parameter constraints report the marginalized median along with the 68\% credible intervals, consistent with the 1D posteriors shown in Figure 15.}
    \resizebox{\textwidth}{!}{
    \begin{tabular}{ll}
        \toprule
        Parameter & Median and 68\% credible interval \\
        \midrule
        $\Omega_m$ & $0.24^{+0.01}_{-0.01}$ \\
        $H_0$ & $72.97^{+0.32}_{-0.30}\,\text{km s}^{-1}\,\text{Mpc}^{-1}$ \\
        $w_0$ & $-0.70^{+0.04}_{-0.06}$ \\
        $w_a$ & $-2.78^{+0.31}_{-0.16}$ \\
        $S_8$ & $0.718^{+0.030}_{-0.030}$ \\
        \bottomrule
    \end{tabular}
    }
    \vspace{0.3em}

    {\footnotesize MAP values are provided in the text for exact replication.}
    \label{tab:results}
\end{table}

\begin{figure}[htbp]
    \centering
    \includegraphics[width=0.75\textwidth]{10_fig_01_iso_p95_maps.png}
    \caption{\textbf{Parameter confidence contours (global scan).}
    Joint constraints highlight a correlation between matter density $\Omega_m$ and the equation of state $w_0$. The likelihood peak (marked by a cross) is located at the exact best-fit (MAP) coordinates ($\Omega_m = 0.243$, $w_0 = -0.69$). Constraints are shown for the cosmological implementation $\Psi$TMG within the overarching MCGT framework.}
    \label{fig:contours}
\end{figure}

\begin{figure}[htbp]
    \centering
    \includegraphics[width=0.9\textwidth]{../output/ptmg_corner_plot.pdf}
    \caption{Marginalized 1D and 2D posterior distributions (68\% and 95\% CL) for the $\Psi$TMG parameters. Titles above the 1D posteriors reflect the marginalized median and 68\% credible intervals. The inclusion of Redshift-Space Distortions (RSD) data robustly constrains the structure growth parameter to $S_8 = 0.718^{+0.030}_{-0.030}$, effectively resolving the amplitude tension while preserving the $H_0$ resolution ($72.97\,\text{km s}^{-1}\,\text{Mpc}^{-1}$).}
    \label{fig:mcmc_corner}
\end{figure}

\clearpage

% --- CHAPTER 11 ---
\subsection*{Chapter 11: LSS Power Spectrum ($S_8$)}
\textbf{Focus: Dark matter and lensing.}
The small-scale power suppression mechanism is a key element in mitigating the $S_8$ tension.

\begin{figure}[htbp]
    \centering
    \includegraphics[width=0.9\textwidth]{11_fig_01_power_comparison.png}
    \caption{\textbf{Matter power spectra comparison.}
    The upper panel shows the spectra for $\Psi$TMG (blue) and $\Lambda$CDM (orange).
    The lower panel (ratio) indicates a power suppression of about 10\% at small scales ($k > 1 h/\mathrm{Mpc}$), consistent with gravitational lensing constraints. \textit{\footnotesize (Data: Planck 2018 Lensing. Script: pipeline/plots.py. Commit: v3.1.0)}}
    \label{fig:power_spectrum}
\end{figure}

\clearpage

% --- CHAPTER 12 ---
\subsection*{Chapter 12: CMB Consistency Check (Acoustic Scale Prior)}
\textbf{Focus: Joint analysis.}
Posterior diagnostic against the Planck 2018 spectra. Note that the full Planck power spectrum likelihood is not directly sampled during the MCMC inference, which relies on the compressed CMB distance prior. The TT/TE/EE residual spectra shown here serve as a posterior diagnostic and sanity check to confirm that the primordial acoustic structure remains undisturbed. The exact best-fit (MAP) point lies within the confidence region.

\begin{figure}[htbp]
    \centering
    \includegraphics[width=0.8\textwidth]{12_fig_cmb_residuals.png}
    \caption{\textbf{CMB temperature spectrum (residuals).} Difference between the $\Psi$TMG theoretical model and Planck 2018 data. Note that the full Planck power spectrum likelihood is not directly sampled during the MCMC inference, which relies on the compressed CMB distance prior. The TT/TE/EE residual spectra shown here serve as a posterior diagnostic and sanity check to confirm that the primordial acoustic structure remains undisturbed. The residuals remain consistent with cosmic noise. \textit{\footnotesize (Data: Planck 2018 TT,TE,EE. Script: pipeline/plots.py. Commit: v3.1.0)}}
\end{figure}

\clearpage
\section{Synthesis: Addressed Tensions and Implications}

The $\Psi$TMG model proposes a unified reading of observational discrepancies, bridging constraints from multiple probes.

\begin{itemize}
    \item \textbf{Hubble Tension ($H_0$):} marginalized median $H_0^{\Psi\mathrm{TMG}} = 72.97^{+0.32}_{-0.30}\,\text{km s}^{-1}\,\text{Mpc}^{-1}$. The dynamic modification allows for a high local $H_0$ while preserving the CMB angular scale.
    \item \textbf{JWST Results:} The potential increase in the early Universe (\autoref{fig:growth}) may contribute to the abundance of massive galaxies at $z>10$.
    \item \textbf{Lensing Tension ($S_8$):} marginalized median $S_8^{\Psi\mathrm{TMG}} = 0.718^{+0.030}_{-0.030}$. The suppression of the high-frequency power spectrum (\autoref{fig:power_spectrum}) reduces the disagreement with Weak Lensing.
\end{itemize}

\begin{figure}[htbp]
    \centering
    \includegraphics[width=0.8\textwidth]{13_fig_tensions_summary.png}
    \caption{\textbf{Tensions summary (whisker plot).} Comparison of $H_0$ and $S_8$ values.
    Top: local measurements (SH0ES) in red and CMB (Planck) in green, in tension.
    Center: the $\Psi$TMG model (blue) overlaps both domains, illustrating a possible statistical reconciliation of the probes.}
    \label{fig:tensions}
\end{figure}

\clearpage

\section{Limitations and Future Work}
\begin{itemize}
    \item \textbf{Dependence on the CPL parametrization:} it is necessary to test other equations of state to confirm that the result is not an artifact of the choice of $w(a)$.
    \item \textbf{Perturbation analysis:} the current study is limited to the linear regime ($k \lesssim 1 h/\mathrm{Mpc}$). Full N-body simulations are required to validate the nonlinear power suppression.
    \item \textbf{Phenomenological nature:} the model is an effective field theory (EFT). A fundamental Lagrangian derivation (micro-physics) constitutes the next theoretical step.
\end{itemize}

\section{Conclusion}
The Metric-Coupled Gravity Theory (MCGT) provides the theoretical geometric framework, while its specific cosmological implementation, denoted $\Psi$TMG, demonstrates data-level improvements on $H_0$, JWST, and $S_8$. Within the scope of the considered datasets and assumptions, $\Psi$TMG stands as a credible candidate for extending the standard model, subject to the discussed limitations and further validations.

\bibliographystyle{plain}
\nocite{*}
\bibliography{references}

\end{document}
